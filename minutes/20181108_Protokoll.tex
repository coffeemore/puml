\begin{center}  
\vspace{15pt}\nointerlineskip\rule{\textwidth}{0.2pt}\\ 
\vspace{0.5pt}\nointerlineskip
\end{center}
\large Datum: 08. November 2018\vspace{3pt}\\\large Protokollant: Patrick Otte
\section{Themen}
\begin{itemize}
\item Aufteilung Präsentation Produktvision
\item Diskussion der möglichen Weiterentwicklung des Produktes
\item offene Fragen/mögliche Technikrecherchen
\end{itemize}
\section{Ergebnisse/Vereinbarungen}
\subsection*{Präsentationsreihenfolge Produktvision:}
\begin{itemize}
\item[1.] Kernfunktionalität
\item[2.] Quick Manual PlantUML
\item[3.] Anwenderspezifikation 
\item[4.] Konkurrenzprodukte
\item[5.] User Story
\item[6.] Datenmodelle
\item[7.] GUI
\item[8.] Mögliche Weiterentwicklungen
\item[9.] Bedarfsermittlung
\item[10.] Workshop Kommunikation
\end{itemize}
\subsection*{Mögliche Weiterentwicklungen:}
\begin{itemize}
\item Plugin für Eclipse
\item Mehrere Sprachen
\item Möglichkeit der Speicherung der Auswahl (Klassendiagramm, Sequenzdiagramm, Methoden
blocken, etc.)
\item Größere Auswahlmöglichkeiten
\item Quick Manual mit ins LaTex Dokument/Manual für den Kunden
\end{itemize}
\section{Hinweise}
\textbf{Freiwillige Zusatzaufgabe:} \\
Erstellung eines PUML Logos
\section*{Nächstes Treffen:}
Donnerstag, 15. November 2018, 8:00 - 9:00 Uhr