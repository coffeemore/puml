\begin{center}  
\vspace{0.5pt}\nointerlineskip\rule{\textwidth}{0.2pt}\\ 
\vspace{0.5pt}\nointerlineskip
\end{center} 
\large Datum: 29.04.2019\vspace{3pt}\\\large Protokollant/in: Geissler,Marian
\section*{Themen}
Sprint 3 Ende \\
Fertige Tickets: 
\begin{itemize}
\item Planung Sequenzdiagramme
\item SeqGen2 
\item Profiler in Betrieb nehmen
\item Fehlersuche und Bugfixing Parser
\item Übergabe fuer XML
\end{itemize}
XML Spezifikationen \\
Beispiele von Oko für: 
\begin{itemize}
\item XML Transformationen für Klassendiagramme
\item XML Transformationen für SQDiagramme
\item Ziel-Ausgabe für SQDiagramme
\item Aktualisiertes UML-Klassendiagramm
Neu: \\
ClassDiagrammGenerator, SequenceDiagrammGenerator (Methode: createDiagramm)
\end{itemize} 
\section*{Ergebnisse/Vereinbarungen}
Start neuer Sprint 4 \\
Ende: 06.05.2019\\
Aufgaben:
\begin{itemize}
\item Ausgabe für Sq. und Klassendiagramme
\item Parser zu XML
\item Generator Klassendiagramme
\item XML Helper Methode
\item Neue Rahmenkonstrukt erstellen
\item UserStories
\end{itemize} 
Tickets bestehend:
\begin{itemize}
\item Gesamtaufruf Test
\item GUI zu AWT/SWING umbauen
\item 
\end{itemize} 
\section*{Hinweise}
Kleinere Funktionen bezueglich XML in neue "XML Helper Methods" implementieren. 
\section*{Nächstes Treffen:}
Freitag, 03.05.2019 7:30 entfällt\\
Montag, 06.05.2019 11:15 (Zwischenstandspräsentation)