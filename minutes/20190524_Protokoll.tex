\begin{center}  
\vspace{0.5pt}\nointerlineskip\rule{\textwidth}{0.2pt}\\ 
\vspace{0.5pt}\nointerlineskip
\end{center} 
\large Datum: 24. Mai 2019\vspace{3pt}\\\large Protokollant/in: Patrick Otte
\section*{Themen}
\subsection*{Review Sprint 5}
\begin{itemize}
\item[•] Vorstellung der Änderungen aufgrund neuer XML-Struktur
\item[•] Vorstellung der neuen Methoden/XMLHelper
\item[•] aktuellen Stand ausgewertet
\end{itemize}
\subsection*{Planning Sprint 6}
\begin{itemize}
\item[•] Endet: 03. Juni 2019
\item[•] Ziel gesetzt (funktionierende Version nach Sprint)
\item[•] neue Tickets erstellt
\begin{itemize}
\item[•] Bugfixes und Feinschliff Sequenzdiagrammgenerator [PUML-92]
\item[•] XML Compare für UnitTests [PUML-93]
\item[•] GUI Auflistung Klassen/Methoden [PUML-94]
\item[•] C++ Parser schreiben [PUML-95]
\item[•] Sequenzdiagramm Output fertigstellen [PUML-96]
\item[•] Test openjdk oracle jdk [PUML-98]
\end{itemize}
\item[•] alte Tickets übernommen
\begin{itemize}
\item[•] alte Codefragmente entfernen [PUML-84]
\item[•] Anpassung Konsole auf neue Methoden [PUML-85]
\item[•] Parser umbauen [PUML-89/90]
\end{itemize}
\item[•] 
\end{itemize}
\section*{Ergebnisse/Vereinbarungen}
\begin{itemize}
\item[•] alten Code entfernen (Siehe Ticket)
\item[•] relative Pfade verwenden
\item[•] Tickets annehmen
\item[•] Zeiten in alte/neue Tickets eintragen
\item[•] PUML Logo bestimmt (siehe Abbildung)
\end{itemize}
\begin{figure}
	\centering
	\includegraphics[width=8cm]{bilderMinutes/PLogo4}
	\caption{PUML Logo Variante 4}
\end{figure}
\section*{Hinweise}
---
\section*{Nächstes Treffen:}
Montag, 27. Mai 2019, 11:15 - 12:45 Uhr