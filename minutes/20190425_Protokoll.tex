\pagebreak
\begin{center}  
\vspace{0.5pt}\nointerlineskip\rule{\textwidth}{0.2pt}\\ 
\vspace{0.5pt}\nointerlineskip
\end{center} 
\large Datum: 25. April 2019\vspace{3pt}\\\large Protokollant/in: Michael Lux
\section{Themen}
Anlegen der parsedData xml Vorlage fuer XML-Struktur
\begin{itemize}

\item Beispiel XML-Datei mit Klassen
\item Beziehungen zwischen Klassen -> fuer Klassendiagramm
\item in Klassen aufgerufene Methoden -> fuer Sequenzdiagramm

\item aus dieser sollen die XML-Dateien fuer die Klassen- und Sequenzdiagramme erzeugt werden, welche beide wiederrum von OutputPUML uebernommen werden (als XML-Dokumente)

\item Nutzen der XML-Struktur: Alle moeglichen zu parsenden Sprachen koennen so einfacher integriert werden 

\item XML-Struktur wird als Dokument und nicht als Filestream gespeichert, um einfacheres parsen dieser zu ermoeglichen.



\end{itemize}


\section{Ergebnisse/Vereinbarungen}
\begin{itemize}
\item Anzahl weiterer geplanter Sprints bis Ende: vorraussichtlich zwei
\end{itemize}
\section{Hinweise}
\begin{itemize}
\item Sprint endet in einer Woche
\item Bis dahin wird auf dem Develop-Branch gearbeitet
\end{itemize}
\section*{Nächstes Treffen:}
Montag, 29. April 2019, 11:15




