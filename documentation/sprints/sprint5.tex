\nsecbegin{Ziel des Sprints}
Ziel des Sprints ist es, die Erzeugung von Klassendiagrammen und Sequenzdiagrammen zu verbessern. Dies soll erreicht werden durch eine Verfeinerung der XML-Strukturspezifizierung der Klassen- und Sequenzdiagramm-Generatoren und durch Erstellung einheitlicher Funktionen zur Traversierung selbiger Struktur, sowie durch Änderungen am Parser selbst. Die Funktionalität des Programms soll durch erweiterte Konsolenbefehle verbessert werden.
\nsecend

\nsecbegin{User-Stories des Sprint-Backlogs}
\nsecbegin{Vorschau der Diagramme}
Als Benutzer wünsche ich mir eine Vorschau der Diagramme, damit ich einschätzen kann ob ich
damit zufrieden bin.
\nsecend
\nsecbegin{Anzeigen und Speichern von PlantUML}
Als Benutzer wünsche ich mir, Diagramme als PlantUML-Code anzeigen und speichern zu können, um den Aufbau nachvollziehen zu können.
\nsecend
\nsecbegin{Erstellung von Klassendiagrammen}
Als Benutzer wünsche ich mir, Klassendiagramme aus meinem bestehenden Quellcode erstellen zu können, damit ich einen Überblick über die Klassen meines Programms und deren Beziehungen bekomme.
\nsecend
\nsecbegin{Generierung von Sequenzdiagrammen}
Als Benutzer wünsche ich mir, Sequenzdiagramme erstellen zu können, um einen Überblick über die Abläufe meines Programms zu erhalten.
\nsecend
\nsecend % {User-Stories des Sprint-Backlogs}

\nsecbegin{Zeitliche Planung}
Der Sprint geht vom 13.05.19 bis zum 24.05.19.
\nsecend%Zeitliche Planung

\nsecbegin{Liste der durchgeführten Meetings}
\begin{itemize}
\item Planning-Meeting (13.05.2019)
\item Zwischen-Meeting (17.05.2019)
\item Review-Meeting (24.05.2019)
\end{itemize}{}
\nsecend%Liste der durchgeführten Meetings

\nsecbegin{Ergebnisse des Planning-Meetings}
Die Zuweisung der Aufgabengebiete fiel auf die Teams zurück, die diese bereits zuvor angegangen sind, da diese mit ihnen vertraut sind und sonst zusätzliche Zeit für die Einarbeitung notwendig wäre.
Es wurde beschlossen, dass die Klassen für die Generierung von Klassen- und Sequenzdiagrammen sowie die Hilfsfunktionen zur Traversierung der XML-Struktur vollständig auf die Verwendung von XPath-Ausdrücken umgestellt werden sollen. 
\nsecend

\nsecbegin{Aufgewendete Arbeitszeit pro Person$+$Arbeitspaket}
\begin{longtable}{|p{4cm}|l|l|l|l|l|}
        \hline
        Arbeitspaket & Person & Start & Ende & h & Artefakt\\
        \hline
        GUI in develop-Branch integrieren & Julian U. & 13.05.2019 & 24.05.2019 & 2 & GUISwing.java \\ \hline
        Generator für Sequenzdiagramme & Leo R. & 13.05.2019 & 24.05.2019 & 19 & SequenceDiagramGenerator.java\\ \hline
        Generator für Sequenzdiagramme & Elisabeth S. & 13.05.2019 & 24.05.2019  & 21 & SequenceDiagramGenerator.java \\ \hline
        Ausgabe für Sequenz-und Klassendiagramme & Tore A.  & 13.05.2019 & 24.05.2019  & 8  & SequenceDiagramGenerator.java \\ \hline
        Ausgabe für Sequenz-und Klassendiagramme & Patrick O.  & 13.05.2019 & 24.05.2019  & 8  & SequenceDiagramGenerator.java \\ \hline
        Generator für Klassendiagramme & Johann G.  & 13.05.2019 & 24.05.2019  & 2 & ClassDiagrammGenerator.java \\ \hline
        Sequenz-Diagramm-Tree und Umbau & Jan S. & 13.05.2019 & 24.05.2019 & 3 & SequenceDiagramGenerator.java \\ \hline
        Alte Codefragmente entfernen  & Marian G. & 13.05.2019 & 24.05.2019 & 2.5 & OutputPUML.java ParserJava.java\\ \hline
        Anpassung Console auf neue Methoden & Marian G. & 13.05.2019 & 24.05.2019  & 6  & Console.java \\ \hline
        Java Parser umbauen & Jona M. & 13.05.2019 & 24.05.2019 & 12 & ParserJava.java \\ \hline
        Java Parser umbauen & Michael L. & 13.05.2019 & 24.05.2019 & 8 & ParserJava.java \\ \hline

\end{longtable}     
\nsecend

\nsecbegin{Konkrete Code-Qualität im Sprint}
Lesbarkeit ist annehmbar aber durchaus noch optimierbar. Immer wieder werden Bugs gefunden und behoben.
\nsecend%Konkrete Code-Qualität im Sprint

\nsecbegin{Konkrete Test-Überdeckung im Sprint}
Die Testüberdeckung ist mit 60,2\% deutlich gestiegen.
\nsecend%Konkrete Test-Überdeckung im Sprint

\nsecbegin{Ergebnisse des Reviews}

\begin{table}[H]

\begin{tabularx}{\textwidth}{ |l|l|X| }
\hline
\textbf{Klasse} & \textbf{Methode} & \textbf{Anmerkungen}\\
 \hline
ClassGeneratorTest.java & allgemein & Testklasse fur Classdiagrammgenerator erstellt, die erzeugtes Diagramm mit Vorlage ClassDiagram.xml vergleicht\\ \hline
SequenzDiagramGenerator.java & allgemein & Funktionen kommentiert, Abfangen von Exceptions  (u.a. XPathExpressionException), alten Code durch XMLHelperMethods wie getChildwithName ersetzt\\ \hline
SequenzDiagramGenerator.java & handleLocalInstances & Funktion eingefügt, die die Klassen-Tags von lokalen Instanzen in das Dokument einfügt\\ \hline
SequenzDiagramGenerator.java & addClassesToInstances & Hinzufügen der Klassen-Tags in methodcalls funktioniert grundlegend (mit kleinen Bugs)\\ \hline
ClassDiagramGenerator.java & createDiagram & Verbesserte Suche nach Vererbungen, Kompositionen, Aggregationen und Implementierungen (funktioniert), alten Code entfernt\\ \hline
GUI\_Swing.java & initialize & GUI auf XML umgestellt und entsprechende Klassen eingebunden, Baumstruktur (JTree) zur Auswahl des Einstiegspunktes überarbeitet\\ \hline
GUI\_Swing.java & initialize & Tree für Sequenzdiagramm erstellt, GUI-Tabulatoren umgestellt\\ \hline
XmlHelperMethods.java & removeComments & Funktion eingefuegt, die Kommentare aus XML-Baum entfernt\\ \hline
XmlHelperMethods.java & removeWhitespace & Funktion eingefügt, die Document als String ohne Leerzeichen zurück gibt\\ \hline
OutputPUML.java & allgemein & Neuimplementation der Erstellung von PUML aus Klassendiagramm mit Verwendung von getList mit Referenzknoten, Umbau der Iteration durch XML Diagramm für Methoddefinitionen\\ \hline
ParserJava.java & buildTree & Grundlegende Struktur für Erkennung von Funktionen angelegt\\ \hline
 
  
 

\hline
\end{tabularx}
\end{table}

Die Ausgabe von Klassendiagrammen funktioniert, es gibt an mehreren Stellen noch Codefehler. Die Grundstruktur wurde ausgearbeitet, es fehlen jedoch noch u.a. die umfangreiche Erkennung von Funktionen.
Beim Review wurden die Änderungen an der XML-Struktur der Klassen- und Sequenzdiagrammen, an den Hilfsfunktionen zur Auswertung selbiger und die neuen Konsolenbefehle vorgestellt, damit alle Teammitglieder mit diesen vertraut sind und mit ihnen arbeiten können.
Für die Änderungen am Parser und an den Klassen- u. Sequenzdiagrammen wurden die Tests erweitert, bzw. neu erstellt. (z.B. ClassDiagram2.xml)

In den Klassen OutputPUML, ParserIF, ParserJava wurde zahlreich nicht mehr benötigter Code entfernt.

\nsecend%Ergebnisse des Reviews

\nsecbegin{Abschließende Einschätzung des Product-Owners}
Mit dem weiter voranschreitenden Umbau werden bald neue Funktionalitäten integriert sowie die bestehenden Funktionalitäten verbessert werden.
\nsecend%Abschließende Einschätzung des Product-Owners

\nsecbegin{Abschließende Einschätzung des Software-Architekten}
Der Umbau ist weitestgehend abgeschlossen. Die neue GUI wurde in den Hauptzweig integriert. Es können wieder Klassendiagramme erzeugt werden, auch wenn diese noch fehlerbehaftet sind.
\nsecend%Abschließende Einschätzung des Software-Architekten