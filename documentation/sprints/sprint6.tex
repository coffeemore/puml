\nsecbegin{Ziel des Sprints}
Ziel des Sprints ist es unter anderem alte Codefragmente zu entfernen, einige Klassen auszubessern und an neue Vorraussetzungen anzupassen und mit dem Erstellen des C++ Parsers zu beginnen. Vor allem sollte am Ende des Sprints das Programm lauffähig sein.
\nsecend

\nsecbegin{User-Stories des Sprint-Backlogs}
\nsecbegin{Vorschau}
Als Benutzer wünsche ich mir eine Vorschau der Diagramme, damit ich einschätzen kann ob ich
damit zufrieden bin.
\nsecend
\nsecbegin{Klassendiagramme}
Als Benutzer wünsche ich mir, Klassendiagramme aus meinem bestehenden Quellcode erstellen zu können, damit ich das nicht manuell tun muss.
\nsecend
\nsecbegin{Anzeigen und Speichern von PlantUML}
Als Benutzer wünsche ich mir, Diagramme als PlantUML-Code anzeigen und speichern zu können, um den Aufbau nachvollziehen zu können.
\nsecend
\nsecend

\nsecbegin{Zeitliche Planung}
Der Sprint geht vom 24.05.19 bis zum 07.06.19.
\nsecend

\nsecbegin{Liste der durchgeführten Meetings}
\begin{itemize}
\item Planning-Meeting (24.05.2019)
\end{itemize}
\nsecend

\nsecbegin{Ergebnisse des Planning-Meetings}
\begin{itemize}
\item Bugfixes und Feinschliff Sequenzdiagrammgenerator [PUML-92]
\item XML Compare für UnitTests [PUML-93]
\item GUI Auflistung Klassen/Methoden [PUML-94]
\item C++ Parser schreiben [PUML-95]
\item Sequenzdiagramm Output fertigstellen [PUML-96]
\item Test openjdk oracle jdk [PUML-98]
\item alte Codefragmente entfernen [PUML-84]
\item Anpassung Konsole auf neue Methoden [PUML-85]
\item Parser umbauen [PUML-89/90]
\item alten Code entfernen
\item PUML Logo bestimmt
\end{itemize}
\nsecend

\nsecbegin{Aufgewendete Arbeitszeit pro Person$+$Arbeitspaket}
\begin{longtable}{|p{4cm}|l|l|l|l|l|}
        \hline
        Parser umbauen & Michael Lux & 24.05.19 & 07.06.19 & 1w 4d 3h & ParserJava.java\\
        \hline
        Parser umbauen & Jona Meyer & 24.05.19 & 07.06.19 & 4d 6h & ParserJava.java\\
        \hline
        Bugfix und Feinschliff von Sequenzdiagrammen & Elisabeth Schuster & 24.05.19 & 07.06.19 & 1d 6h 15m & SequenzDiagramGenerator.java\\
        \hline
        SeqDiagram1 & Leonie Rauschke & 24.05.19 & 07.06.19 & 5h 50m & SequenzDiagramGenerator.java\\
        \hline 
        XML-Compare für Unit-Tests & Patrick Otte & 24.05.19 & 05.06.19 & 2d & \\
        \hline
        Logger Logdatei & Patrick Otte & 24.05.19 & 03.06.19 & 3h &\\
        \hline
        getList Methode & Patrick Otte & 26.05.19 & 27.05.19 & 6h & \\
        \hline\\
        Anpassung Console auf neue Methoden & Marian Geißler & 24.05.19 & 07.06.19 & 1d 6h 30m & \\
        \hline
        GUI auflistung der Klassen und Methoden & Julian Uebe & 24.05.19 & 07.06.19 & 1w 7h & \\
        \hline
        C++-Parser erstellen & Jan Sollmann & 24.05.19 & 07.06.19 & 1w 3d 1h 30m  & ParserCPP.java\\
        \hline
        C++-Parser erstellen & Johann Gerhardt & 24.05.19 & 07.06.19 & 4d 1h  & ParserCPP.java\\
        \hline
        Sequenzdiagramm-Output fertigstellen & Tore Arndt & 24.05.19 & 07.06.19 & 2d 4h & \\
        \hline   
        Test für Sequenzdiagramm Output schreiben & Patrick Otte & 24.05.19 & 27.05.19 & 4h & \\
        \hline     
\hline
\end{longtable}     
\nsecend

\nsecbegin{Konkrete Code-Qualität im Sprint}
Alte und unbenötigte Codefragmente wurden entfernt. In einigen Klassen wurden Bugs behoben. Die Code-Qualität hat sich daher wahrscheinlich verbessert.
\nsecend%Konkrete Code-Qualität im Sprint

\nsecbegin{Konkrete Test-Überdeckung im Sprint}
Es wurde eine Methode für das Vergleichen von XML-Dokumenten geschrieben. Mit dieser lassen sich beispielsweise der ClassDiagramGenerator und der SequenzDiagramGenerator testen. Daher sind in diesem Sprint neue Tests hinzugekommen, wodurch die Testabdeckung hoch gehalten wird.
Insgesamt ist die Testüberdeckung mit 63,5\% leicht gestiegen und im annehmbaren Bereich.
\nsecend%Konkrete Test-Überdeckung im Sprint

\nsecbegin{Ergebnisse des Reviews}
\begin{table}[H]

\begin{tabularx}{\textwidth}{ |l|l|X| }
\hline
\textbf{Klasse} & \textbf{Methode} & \textbf{Anmerkungen}\\
\hline
\end{tabularx}
\end{table}

\nsecend%Ergebnisse des Reviews

\nsecbegin{Ergebnisse der Retrospektive}
Der Sprint lief insgesamt betrachtet erfolgreich. Allerdings wird der CPP-Parser wohl nicht die gleiche Funktionalität wie der Java-Parser erreichen. Das Parsen von CPP-Quellcode funktioniert zwar in Ansätzen, ist allerdings weder vollständig noch für komplexeren Code ausgelegt.
\nsecend%Ergebnisse der Retrospektive

\nsecbegin{Abschließende Einschätzung des Product-Owners}
XXX
\nsecend%Abschließende Einschätzung des Product-Owners

\nsecbegin{Abschließende Einschätzung des Software-Architekten}
Der Java-Parser nimmt langsam Form an. Da sich die für das Projekt vorhandene Zeit dem Ende naht, wurden folgende Ziele für den Projektabschluss vereinbart:
\begin{itemize}
\item Klassendiagramme sollen aus jedem Java-Code generiert werden können
\item Sequenzdiagramme sollen zumindest für die Java-Spezifikation funktionieren
\item Klassendiagramme sollen auch für die C++-Spezifikation funktionieren
\end{itemize}
\nsecend%Abschließende Einschätzung des Software-Architekten

\nsecbegin{Abschließende Einschätzung des Team-Managers}
XXX
\nsecend%Abschließende Einschätzung des Team-Managers