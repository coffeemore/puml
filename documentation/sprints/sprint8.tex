\nsecbegin{Ziel des Sprints}

\nsecend

\nsecbegin{User-Stories des Sprint-Backlogs}

\nsecend % {User-Stories des Sprint-Backlogs}

\nsecbegin{Zeitliche Planung}

\nsecend%Zeitliche Planung

\nsecbegin{Liste der durchgeführten Meetings}

\nsecend%Liste der durchgeführten Meetings

\nsecbegin{Ergebnisse des Planning-Meetings}

\nsecend

\nsecbegin{Aufgewendete Arbeitszeit pro Person$+$Arbeitspaket}

\nsecend

\nsecbegin{Konkrete Code-Qualität im Sprint}
Hat sich zum vorhergehenden Sprint kaum verändert.
\nsecend%Konkrete Code-Qualität im Sprint

\nsecbegin{Konkrete Test-Überdeckung im Sprint}
Die Testüberdeckung ist mit 62,3\% leicht zurückgegangen, aber noch durchaus im vertretbaren Bereich.
\nsecend%Konkrete Test-Überdeckung im Sprint

\nsecbegin{Ergebnisse des Reviews}

\nsecend%Ergebnisse des Reviews

\nsecbegin{Abschließende Einschätzung des Product-Owners}
Insgesamt konnte das Produkt zu einem zufriedenstellenden Ausmaß finalisiert werden. Außer einzelner Bedienungseigenarten der GUI funktioniert das Erstellen von Klassen- und Sequenzdiagrammen für die definierten Testfälle intuitiv.
\nsecend%Abschließende Einschätzung des Product-Owners

\nsecbegin{Abschließende Einschätzung des Software-Architekten}
\begin{itemize}
\item Die nach Sprint 6 gesetzten Ziele wurden erreicht.
\item Der Logger ist weitestgehend integriert.
\item Die Kommandozeilenschnittstelle ist umfangreich.
\item Die GUI und das Betrachten der Diagramme darin funktioniert
\item Es sind aber noch einige kleinere Bugs und Unschönheiten vorhanden
\end{itemize}
\nsecend%Abschließende Einschätzung des Software-Architekten