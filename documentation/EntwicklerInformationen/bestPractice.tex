\nsecbegin{Best practice}
\nsecbegin{GIT}
\nsecbegin{Keine nicht benötigten Dateien adden}
Vor dem "'git add ."' immer mit "'git status"' prüfen welche Dateien hinzugefügt werden. Sollten nicht für das Projekt benötigte Dateien (z.B. übersetzte Binärdateien oder Dokumentation von Librarys) dabei sein, bitte die entsprechende "'.gitignore-Datei"' vervollständigen. Danach sollten die Dateien beim "'git status"' nicht mehr angezeigt und somit nicht mehr geadded werden.\\
HINWEIS: Die "'.gitignore-Datei"' ist (wie an dem führenden Punkt zu sehen ist) versteckt und wird nur nach dem setzen des entsprechenden Häckchens im Dateimanager oder beim "'ls -a"' angezeigt.
\nsecend
\nsecbegin{Branches nach aktuell zu bearbeitendem Thema benennen}
Damit direkt aus dem Branchname ersichtbar ist was innerhalb des Branches bearbeitet wird, ist es sinnvoll den Name entsprechend zu wählen (Z.B. GUIBranch). Da die Branches mit dem Gitlab-Server synchronisiert werden können, ist es auch ohne weiteres möglich dass mehrere Personen an einem Branch arbeiten.
\nsecend
\nsecend
\nsecend