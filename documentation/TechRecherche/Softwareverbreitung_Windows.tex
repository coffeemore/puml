\UseRawInputEncoding
\nsecbegin{Softwareverbreitung unter Windows}
	\begin{itemize}
	\item Beschreibung
	
	
	Um eine unbeaufsichtigte Installation von Software zu gew�hrleisten ben�tigt unser Projekt f�r Windows ein installierbares Softwarepaket. Um ein solches Softwarepaket zu erstellen, ben�tigt man die Hilfe eines Installers.
	
	\item Installer-Typen
	
	
    \begin{itemize}
	\item Kriterien
	
	
	Kriterien bez�glich der Installer sind zum einen, ob selbige mit Open-Source arbeiten. Zudem sollten die Installer im Bezug auf das nicht-kommerzielle Projekt nicht kostenpflichtig sein. Im Generellen sollte der ideale Installer au�erdem eine einfache Handhabung innehaben und im Spezifischen die �berpr�fung, ob die JRE installiert ist, und gegebenenfalls die Installation derselbigen anbieten.
	
	\item Nullsoft Scriptable Install System (NSIS)
	
	
	\textbf{NSIS} bietet ein kostenlosen aber sehr flexiblen wie auch minimalen Installer. Jedoch sind durch Open-Source bereits viele Plugins vorhanden. Auch hier ist eine �berpr�fung und Installation der JRE m�glich. Der Compiler der NSIS's ist jedoch recht primitiv gestaltet und ein intuitives Verst�ndnis des Codes gestaltet sich durch fehlendes Syntax-Highlighting schwer.
	
	\item Inno Setup
	
	
	Das \textbf{Inno Setup} ist ein Open-Source-unterst�tzender Installer welcher kostenlos downloadbar ist. Der Installer kommt mit einem �bersichtlichen Compiler, einfacher und gut strukturierter Syntax, wie einer Code-Sektion in welcher komplexe Vorg�nge mit Pascal programmiert werden k�nnen, und zudem mit vorgefertigten Beispielen zu bestimmten Software-Typen. Auch eine �berpr�fung und eventuelle Installation der JRE ist umsetzbar. 
	\end{itemize}
	\item Fazit
	
	
	Nach Betrachtung beider Installer gen�gen beide Typen vollkommen den gestellten Anforderungen. Jedoch ist das \textit{Inno Setup} durch �bersichtlichen Compiler und intuitiver Handhabung meine pers�nliche Empfehlung im Sinne des Softwareprojekts.
	\end{itemize}

\nsecend


