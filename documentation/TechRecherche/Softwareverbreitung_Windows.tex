%\UseRawInputEncoding
\nsecbegin{Softwareverbreitung unter Windows (Autor: Jan Sollmann)}
	\begin{itemize}
	\item Beschreibung
	
	
	Um eine unbeaufsichtigte Installation von Software zu gewährleisten benötigt unser Projekt für Windows ein installierbares Softwarepaket. Um ein solches Softwarepaket zu erstellen, benötigt man die Hilfe eines Installers.
	
	\item Installer-Typen
	
	
    \begin{itemize}
	\item Kriterien
	
	
	Kriterien bezüglich der Installer sind zum einen, ob selbige mit Open-Source arbeiten. Zudem sollten die Installer im Bezug auf das nicht-kommerzielle Projekt nicht kostenpflichtig sein. Im Generellen sollte der ideale Installer außerdem eine einfache Handhabung innehaben und im Spezifischen die Überprüfung, ob die JRE installiert ist, und gegebenenfalls die Installation derselbigen anbieten.
	
	\item Nullsoft Scriptable Install System (NSIS)
	
	
	\textbf{NSIS} bietet ein kostenlosen aber sehr flexiblen wie auch minimalen Installer. Jedoch sind durch Open-Source bereits viele Plugins vorhanden. Auch hier ist eine Überprüfung und Installation der JRE möglich. Der Compiler der NSIS's ist jedoch recht primitiv gestaltet und ein intuitives Verständnis des Codes gestaltet sich durch fehlendes Syntax-Highlighting schwer.
	
	\item Inno Setup
	
	
	Das \textbf{Inno Setup} ist ein Open-Source-unterstützender Installer welcher kostenlos downloadbar ist. Der Installer kommt mit einem übersichtlichen Compiler, einfacher und gut strukturierter Syntax, wie einer Code-Sektion in welcher komplexe Vorgänge mit Pascal programmiert werden können, und zudem mit vorgefertigten Beispielen zu bestimmten Software-Typen. Auch eine Überprüfung und eventuelle Installation der JRE ist umsetzbar. 
	\end{itemize}
	\item Fazit
	
	
	Nach Betrachtung beider Installer genügen beide Typen vollkommen den gestellten Anforderungen. Jedoch ist das \textit{Inno Setup} durch übersichtlichen Compiler und intuitiver Handhabung meine persönliche Empfehlung im Sinne des Softwareprojekts.
	\end{itemize}

\nsecend


