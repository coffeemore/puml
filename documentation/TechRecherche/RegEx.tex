\nsecbegin{Reguläre Ausdrücke}
Reguläre Ausdrücke sind Beschreibungen eines Musters, sog. Patterns, die bei Zeichenkettenverarbeitung eingesetzt werden. Mittels dieser Muster lassen sich Zeichenketten suchen und ersetzen.\\
\nsecbegin{Funktionen}
\begin{itemize}
\item[(1)] Komplette Übereinstimmung suchen \\
\begin{tabular}{l c}
\begin{lstlisting}
Pattern.matches(regex, this);
\end{lstlisting} & \begin{lstlisting}
Pattern p = Pattern.compile(regex);
Matcher m = p.matcher(input);
return m.matches();
\end{lstlisting} \\
\end{tabular}
\item[(2)] Teilstring finden
\begin{itemize}
\item[$\bullet$] alle Vorkommen des Teilstrings innerhalb eines Suchstrings suchen
\end{itemize}
\item[(3)] Teilfolgen ersetzen
\item[(4)] Zerlegen einer Zeichenfolge
\begin{itemize}
\item[$\bullet$] Trennzeichen sind durch Muster definiert, resultiert in Sammlung von Zeichenfolgen
\end{itemize}
\end{itemize}
\nsecend
\nsecbegin{Verwendung}
\begin{itemize}
\item[]Um mit Regulären Ausdrücken arbeiten zu können, wird das Paket ‚java.util.regex‘ implementiert. Es enthält die Klassen Matcher (Zugriff auf Mustermaschine) und Pattern (Repräsentation RE in vorkompiliertem Format).
Außerdem gibt es verschieden Klassifizierungen, um die Suche genauer zu definieren.
\end{itemize}
%Tabelle1
\begin{table} [H]
\centering
\begin{tabular}{l|c}
\multicolumn{1}{l}{\textbf{Quantifizierung}} & \textbf{Anzahl der Wiederholungen}\\
\hline
X? & X kommt einmal oder keinmal vor \\
\hline
X* & X kommt keinmal oder beliebig oft vor \\
\hline
X+ & X kommt einmal oder beliebig oft vor \\
\hline
X\{n\} & X muss genau n-mal vorkommen \\
\hline
X\{n,\} & X kommt mindestens n-mal vor \\
\hline
X\{n,m\} & X kommt mindestens n-, aber max. m-mal vor \\
\hline
\end{tabular}
\end{table}
%Tabelle2
\begin{table} [H]
\centering
\begin{tabular}{l|c}
\multicolumn{1}{l}{\textbf{zeichenklasse}} & \textbf{Enthält}\\
\hline
. & jedes Zeichen \\
\hline
[aei] & Zeichen a, e, i \\
\hline
[\^{}aei] & nicht die Zeichen a, e, i \\
\hline
[0-9a-f] & Zeichen 0-9 oder Kleinbuchstaben a-f \\
\hline
\textbackslash d & Ziffer: [0-9] \\
\hline
\textbackslash D & keine Ziffer: [\^{}0-9] bzw. [\^{}\textbackslash d] \\
\hline
\textbackslash p\{Blank\} & Leerzeichen oder Tab: [\textbackslash t] \\
\hline
\textbackslash p\{Lower\}, \textbackslash p \{Upper\} & Klein-/Großbuchstaben: [a-z] bzw. [A-Z] \\
\hline
\end{tabular}
\end{table}
weitere Klassifizierungen: \\
\href{https://docs.oracle.com/javase/7/docs/api/java/util/regex/Pattern.html}{https://docs.oracle.com/javase/7/docs/api/java/util/regex/Pattern.html}
\nsecend
\nsecbegin{Beispiele}
\begin{itemize}
\item[(1)] Rückgabewert beider Abfragen ist true
\begin{lstlisting}
System.out.println(Pattern.matches("'.*'","'Hallo Welt'" ));
System.out.println("'Hallo Welt'".matches("'.*'"));
\end{lstlisting}
\item[(2)] Abfrage nach Teilstring, Rückgabewert ist gefundener Teilstring
\begin{lstlisting}
String text = "Moderne Programmiersprachen haben durch die Vernetzung von Computern neue Anforderungen erfahren. So lautet auch ein Motto von Sun: 'The Network is the Computer.' ";
Matcher matcher = Pattern.compile("'.*'").matcher(text);
while(matcher.find()) {
	System.out.println(matcher.group());
}
\end{lstlisting}
\end{itemize}
\nsecend
\nsecend