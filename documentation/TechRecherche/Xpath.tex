\nsecbegin{XPath (Autor: Michael Lux)}
Abfragesprache zur Addressierung/Auswertung von XML und Grundlage für Standards: XLST, XPointer, XQuery.
\nsecbegin{Aufbau und Verwendung}
\begin{itemize} 
	
	\item XML-Dokument wird als Baum betrachtet
	\begin{itemize}
		\item Knoten (nodes): Dokumenten-Knoten, XML-Elemente, -Attribute, -Textknoten, -Kommentare, -Namensräume und -Verarbeitungsanweisungen
		\item Achsen: preceding, following, preceding-sibling und following-sibling	
	\end{itemize}

\end{itemize}

\begin{itemize} 
	
	\item XPath-Ausdruck besteht aus mehreren Lokalisierungsschritten:
	\begin{itemize}
		\item achse::knotentest[prädikat 1][prädikat 2]...
		\item Beispiel: /descendant-or-self::Foo
		\item Prädikate: Funktionen/Operatoren zur weiteren Einschränkung
		\item Beispiel: text(), comment() für bestimmten Datentyp	
	\end{itemize}

\end{itemize}
\nsecend
\nsecbegin{Beispiel an dargestellter XML-Datei}
\begin{itemize}
	\item /descendant-or-self::Softwareprojekt bzw. Softwareprojekt
	
	Wählt alle untergeordnete Knoten inklusive des Kontextknotens Softwareprojekt aus.
	
	\item /descendant-or-self::Softwareprojekt/descendant::Gruppe bzw. Softwareprojekt//Gruppe
	
	Wählt alle untergeordnete Knoten von Gruppe aus.
	
	\item /descendant-or-self::Softwareprojekt/descendant::Gruppe/descendant::Master/descendant::Projektowner/attribute::* oder /Softwareprojekt/Gruppe/Master/Projektowner/attribute::*
	
	liefert Attribute='mID=1' zurück.
	
\end{itemize}
\nsecend
\nsecend