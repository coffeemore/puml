\nsecbegin{GUI}

\nsecbegin{Welche GUI-Frameworks gibt es für Java?\\[12pt]}
Aktuell gibt es folgende Möglichkeiten zur Erstellung einer GUI Oberfläche in Java:
 \begin{itemize}
	\item Swing
	\item JavaFX
	\item Standard Widget Toolkit (SWT)
	\item Abstract Window Toolkit (AWT)
	\item Google Web Toolkit (GWT)
	\item Qt (Qt Jambi)
	\item GTK+
\end{itemize}
\nsecend %{Welche GUI-Frameworks gibt es für Java?}

\nsecbegin{Welche können für das Projekt genutzt werden?\\}
Für unser Projekt kommen aktuell die Frameworks Swing und JavaFX in Frage. Zum Einen sind beide aktuell die beiden meist genutzten GUI Frameworks in Java, zum Anderen sind hierfür Eclipse Plugins verfügbar.
\nsecend %{Welche können für das Projekt genutzt werden?}

\nsecbegin{Vor- und Nachteile der Frameworks}

\nsecbegin{Swing\\}

\textbf{Vorteile}
\begin{itemize}
	\item Bestandteil des Java Development Kits/Java Foundation Classes
	\item Nutzt eine Sammlung von Bibliotheken zur GUI-Programmierung (Bspw. AWT)
\end{itemize}


\textbf{Nachteile}
\begin{itemize}
	\item Wird nicht mehr weiterentwickelt oder gewartet
	\item Probleme im Bereich Medieneinbindung und Animation
	\item Bestimmte Anwendung wie bspw. Zooming nicht möglich
\end{itemize}

Swing ist eines der meistgenutzten GUI-Frameworks für Java, war bis 2014 ein Standard-Tool zur GUI Entwicklung und hat aufgrund dessen eine große Community hinter sich und man findet viel Hilfestellungen für Swing im Internet.

\nsecend %{Swing}

\nsecbegin{JavaFX\\}

\textbf{Vorteile}
\begin{itemize}
	\item Teil jeder neuen Java SE Installation
	\item Möglichkeit einfach animierte Übergänge einzubinden
	\item optisch ansprechender
	\item Anwendung kann mittels CSS bearbeitet werden (durch Einbindung von FXML-Code)
\end{itemize}


\textbf{Nachteile}
\begin{itemize}
	\item weniger Online-Hilfe, kleinere Community
\end{itemize}


Aufgrund der erst kurzen Zeit, in der JavaFX zur Verfügung steht, gibt es hier viel weniger Hilfe online im Vergleich zu Swing. JavaFX gilt allerdings als der neue Standard in der Java GUI Entwicklung und es gibt sehr viele Developer, die von Swing zu JavaFX umsteigen.

\nsecend %{JavaFX}

\nsecend %{Vor- und Nachteile der Frameworks}

\nsecbegin{Fazit}

In Hinsicht auf die Langlebigkeit bzw. der Zukunftssicherheit, der moderneren Optik, der Einbindung in Eclipse und dem steigenden Support haben wir uns für JavaFX zur GUI Entwicklung in unserem Projekt entschieden.

\nsecend %{Fazit}
\nsecend %{GUI}