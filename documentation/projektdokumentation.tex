\documentclass[twoside]{report}

% ------
% Umlaute
\usepackage{ifluatex,ifxetex}
\ifluatex
  \usepackage{fontspec}
\else
  \ifxetex
    \usepackage{fontspec}
  \else
    \usepackage{selinput}
    \SelectInputMappings{
      adieresis={ä},
      germandbls={ß},
    }
    \usepackage[T1]{fontenc}
    %\usepackage{textcomp}% optional
    %\usepackage{lmodern}
  \fi
\fi

% ------
% Paper auf Deutsch
\usepackage[ngerman]{babel}

% ------
% Page layout
\usepackage[hmarginratio=1:1,top=32mm,columnsep=20pt]{geometry}
\usepackage[font=it]{caption}
\usepackage{paralist}
%\usepackage{multicol}


% ------
% Abstract
\usepackage{abstract}
	\renewcommand{\abstractnamefont}{\normalfont\bfseries}
	\renewcommand{\abstracttextfont}{\normalfont\small\itshape}


% ------
% Titling (section/subsection)
\usepackage{titlesec}
\renewcommand\thesection{\Roman{section}}
\titleformat{\section}[block]{\Large\scshape\bfseries}{\thesection.}{1em}{}
\setcounter{secnumdepth}{3}

% ------
% Tabellen über Seitenumbrüche hinweg
\usepackage{longtable}

%
% Erweiterte Tabellen
\usepackage{tabularx}
\usepackage{multirow}
\usepackage{float}

%Unterschiedliche Zielgruppen des Dokuments
\usepackage{multiaudience}
\SetNewAudience{developer}
\SetNewAudience{manualDE}


% ------
% Header/footer
\usepackage{fancyhdr}
	\pagestyle{fancy}
	\fancyhead{}
	\fancyfoot{}
\begin{shownto}{-, developer, manualDE}
	\fancyhead[C]{Projektdokumentation $\bullet$ PUML $\bullet$ WS18/19$+$SS19}
	\fancyfoot[RO,LE]{}
\end{shownto}
\begin{shownto}{developer}
	\fancyhead[C]{Entwicklerdokumentation $\bullet$ PUML}
	\fancyfoot[RO,LE]{}
\end{shownto}
\begin{shownto}{manualDE}
	\fancyhead[C]{Benutzerhandbuch $\bullet$ PUML}
	\fancyfoot[RO,LE]{}
\end{shownto}


% ------
% Clickable URLs (optional)
%\usepackage{hyperref}
\usepackage[colorlinks=true,linkcolor=black]{hyperref} %Verlinkung des Inhaltsverzeichnisses ohne roten Rahmen um den Text

% ------
% Literaturverweise mit Bibtex einbinden
\usepackage[authoryear,sectionbib,round]{natbib}

% ------
% Bilder laden
\usepackage[pdftex]{graphicx}

\begin{shownto}{-, developer, manualDE}
% ------
% Maketitle metadata
\title{\vspace{-5mm}%
	\fontsize{24pt}{10pt}\selectfont
	\textbf{Projektdokumentation}
	}	
\author{%
        % alle Autoren hier listen
        % 
	\large
	\textsc{Softwarearchitekt: Philipp Rimmele -- philipp.rimmele@stud.htwk-leipzig.de} \\[2mm]
	\textsc{Product Owner: Anna Heinrich -- anna.heinrich@stud.htwk-leipzig.de} \\[2mm]
	\textsc{Marian Geißler -- marian.geissler@stud.htwk-leipzig.de} \\[2mm]
	\textsc{Patrick Otte -- patrick.otte@stud.htwk-leipzig.de} \\[2mm]
	\textsc{Johann Gerhardt -- johann.gerhardt@stud.htwk-leipzig.de} \\[2mm]
	\textsc{Michael Lux -- michael.lux@stud.htwk-leipzig.de}	\\[2mm]	
	\textsc{Jan Sollmann -- jan.sollmann@stud.htwk-leipzig.de} \\[2mm]
	\textsc{Julian Uebe -- julian.uebe@stud.htwk-leipzig.de} \\[2mm]
	\textsc{Elisabeth Schuster -- elisabeth.schuster@stud.htwk-leipzig.de} \\[2mm]		
	\textsc{Jona Meyer}	\\[2mm]	
	\textsc{Leo Rauschke}	\\[2mm]
	\textsc{Tore Arndt -- tore.arndt@stud.htwk-leipzig.de} \\[2mm]
	\normalsize	HTWK Leipzig 
	}
\date{}
\end{shownto}
\begin{shownto}{developer}
\title{\vspace{-5mm}%
	\fontsize{24pt}{10pt}\selectfont
	\textbf{Entwicklerdokumentation}
	}	
\author{%
        % alle Autoren hier listen
        % 
	\large
	\textsc{PUML} \\[2mm]
	}
\date{}
\end{shownto}
\begin{shownto}{manualDE}
\title{\vspace{-5mm}%
	\fontsize{24pt}{10pt}\selectfont
	\textbf{Benutzerhandbuch}
	}	
\author{%
        % alle Autoren hier listen
        % 
	\large
	\textsc{PUML} \\[2mm]
	}
\date{}
\end{shownto}


% Erweiterte Tabellen
\usepackage{tabularx}
\usepackage{multirow}
\usepackage{float}

%Tabellen mit Items
\usepackage{booktabs}% http://ctan.org/pkg/booktabs
\newcommand{\tabitem}{~~\llap{\textbullet}~~}

% Code Beispiel Java
\usepackage{listings}
\lstset{
	language=Java,
	aboveskip=2mm,
	belowskip=2mm,
	showstringspaces=false,
	columns=flexible,
	basicstyle={\small\ttfamily},
	numbers=none,
	breaklines=true,
	breakatwhitespace=true,
	tabsize=3
}

%Drehen von Text
\usepackage{rotating}
\newcommand\tabrotate[1]{\begin{turn}{90}\rlap{#1}\end{turn}}

% Geschachtelte Sections
\newcounter{sectionlevel}

\newcommand{\nsecbegin}[1]
{
    \ifnum\value{sectionlevel} = 0
    \PackageError{nsection}{Die minimale verschachtelungstiefe wurde unterschritten}
    %\part{#1}%
    \else
    \ifcase\value{sectionlevel} 
    \chapter{#1}
    \or
    \section{#1}
    \or
    \subsection{#1}
    \or
    \subsubsection{#1}
    \or
    \paragraph{#1}
    \or
    \subparagraph{#1}
    \else
    \PackageError{nsection}{Die maximale verschachtelungstiefe wurde ueberschritten}
    \fi
    \fi
    \addtocounter{sectionlevel}{1}
}

\newcommand{\nsecend}
{
  \addtocounter{sectionlevel}{-1}
}

\newcommand{\nsecdocumentend}
{
	\ifnum\value{sectionlevel} = 1
	\else
	\PackageError{nsection}{Es fehlt eine nsecend-Anweisung}
	\fi
}

\setcounter{sectionlevel}{1}






%%%%%%%%%%%%%%%%%%%%%%%%
\begin{document}


% -------
% Titel und Abstract über beide Spalten
%\twocolumn[
%\begin{@twocolumnfalse}

\maketitle
\thispagestyle{fancy}

\tableofcontents

%%%%
%%%% Die Struktur des Dokuments bitte nicht aendern!!!
%%%%

\begin{shownto}{-, developer, manualDE}

\nsecbegin{Anforderungsspezifikation}

\nsecbegin{Initiale Kundenvorgaben}
\begin{itemize}
\item Zielbestimmung

PlantUML ist eine einfache, menschen-lesbare Speifikationssprache für die Erzeugung von UML-Diagrammen. Häufig werden allerdings auch UML-Diagramme zu bestehendem Quellcode benötigt, was durch eine gut unterstützte Generierung von PlantUML-Beschreibungen aus Java-Quelltext ermöglicht werden soll.

\item Produkteinsatz
 
Für den Einsatz in der Lehre ist die Generierung von Klassendiagrammen und Sequanzdiagrammen notwendig.

\item Produktbeschreibung
\begin{itemize}
\item eine jar-Datei oder eine Menge an java-Dateien ist einlesbar
\item der Quelltext ist zu analysieren und die identifizierten Klassen und Verknüpfungen sind anzuzeigen - einschließlich use-Beziehungen
\item der Nutzer kann entscheiden, welche Bestandteile in ein Klassen-Diagramm eingehen sollen
\item eine Voransicht des beschriebenen Diagramms ist zu integrieren 
\item eine Unterstützung für das Layout ist ebenfalls anzubieten
\item ferner soll für den Aufruf einer oder mehrerer Methoden ein Sequanzdiagramm abgeleitet werden - auch hier soll der Nutzer die Möglichkeit haben, einzelne tiefere Aufrufe zu blockieren bzw. sich bei alternativen Pfaden für einen Pfad zu entscheiden
\end{itemize}
\end{itemize}
\nsecend

% Hier eure Recherchen für den Produktvisions-Workshop ergänzen! Kopiert den Inhalt eures Dokuments einfach in die bereits dafür angelegte Datei. Ihr könnt auch gern selbst eine erstellen, denkt nur daran, die Aufgabenstellung aus Jira zu kopieren, damit am Ende alles schön einheitlich ist.

\nsecbegin{Produktvision}
\nsecbegin{Kernfunktionalität}
\begin{itemize}
\item Einlesen einer Jar-Datei oder mehrerer Java Dateien
\item Analyse des Java-Source Codes und Identifikation seiner verbundenen Klassen sowie deren Verknüpfungen und Methoden
\item Möglichkeit der Ausgabe eines Klassendiagrams oder eines Sequenzdiagramms
\item Sequenzdiagramm:
\begin{itemize}
	\item Möglichkeit der Ausgabe eines Sequenzdiagramms
	\item Möglichkeit Aufrufe von Methoden im Sequenzdiagramm zu blockieren
\end{itemize}	
\item Klassendiagramm:
\begin{itemize}
	\item Möglichkeiten des Nutzers der Auswahl der Bestandteile in einem Klassendiagramm
	\item Möglichkeit der Voransicht des Klassendiagramms
\end{itemize}	
\item Überstützung für das Layout:
\begin{itemize}
	\item Layout muss automatisch konfigurierbar sein
	\item Layout muss die Möglichkeit haben manuell konfiguriert werden zu können
\end{itemize}	
\item Beide Diagrammarten sollen als String oder als Textdatei ausgegeben werden können
\end{itemize}

Aufgaben für den 1. Sprint:
\begin{itemize}
\item Erstellen eines ersten GUI’s
\item Einlesen der Dateien sowie deren Analyse
\item Ausgabe des Klassendiagramms
\end{itemize}

%Denkanstöße:
%\begin{itemize}
%\item Was soll das Programm auf jeden Fall leisten?
%\item Was sind bezogen auf die Funktionalität die Prioritäten im ersten Sprint?
%\end{itemize}
%Die Kernfunktionalität lässt sich weitgehend aus der Projektbeschreibung ableiten. Ergänzend dazu sollte überlegt werden, ob sich aus dieser Beschreibung noch ungeklärte Fragen ergeben. Die Recherche dient also der Formulierung eines klaren Anforderungsprofils für unser Produkt.
\nsecend
%Autor: Patrick Otte
\nsecbegin{Anwenderprofil}
Denkanstöße:
\begin{itemize}
\item Wer ist unsere Zielgruppe?
\item Worauf legt unsere Zielgruppe besonderen Wert?
\item Was für einen Mehrwert bietet unser Produkt der Zielgruppe?
\end{itemize}
Hier geht es einerseits um eine klare, wenn auch triviale Einordnung, wer unser Produkt später nutzen soll. Andererseits soll erörtert werden, wie das Produkt unserer Zielgruppe das Leben leichter machen kann.
\nsecend

    
    
    



\nsecbegin{PlantUML-Vorstellung}
Denkanstöße:
\begin{itemize}
\item Wie beschreibt man Klassen allgemein mit PlantUML?
\item Welche Arten von Beziehungen zwischen Klassen unterscheidet PlantUML (was für Arten von Pfeilen gibt es)?
\item Gibt es Konstrukte, die mit PlantUML vielleicht nur unzureichend dargestellt werden könnten?
\item Gibt es PlantUML-Erweiterungen, die von besonderem Interesse für unser Projekt sind?
\end{itemize}
Diese Recherche soll einen ersten Einblick in den Aufbau und die Möglichkeiten von PlantUML gewähren.
\nsecend



\nsecbegin{Konkurrenzprodukte (Autor: Julian Uebe)}
%Denkanstöße:
%\begin{itemize}
%\item Gibt es schon ähnliche Ansätze?
%\item Wie unterscheiden / gleichen sich diese bezogen auf ihre Funktionalität?
%\item Welche Nischen gibt es?
%\end{itemize}
%Hier sollen sowohl Marktlücken als auch "Best Practices" identifiziert werden.
\begin{itemize}
	\item ähnliche Ansätze:
	\begin{itemize}
		\item bisher nur z.B. Umwandlung von PlantUML-Code zu Quelltext oder XMI-Dateien
		\item Eclipse-Plugin zum Anzeigen von PlantUML-Code aber nicht zum Generieren
	\end{itemize}
	\item Object-Aid:
		\begin{itemize}
			\item Plugin für Eclipse zum Generieren von UML-Diagrammen aus Quelltext
			\item keine Unterstützung von PlantUML
			\item Drag \& Drop möglich
			\item nicht Open Source
			\item Klassendiagramme sind frei zugänglich
			\item Sequenzdiagramme benötigen kostenpflichtige Lizenz
		\end{itemize}
	\item hpp2plantUML:
		\begin{itemize}
			\item quelloffener Konverter von C++-Qellcode zu PlantUML-Code
		\end{itemize}
\end{itemize}
Fazit:
Es gibt einige Ansätze, die bereits teilweise unsere geplanten Funktionalitäten implementieren.
Letztendlich lässt sich aber sagen, dass es noch kein Projekt gibt, was im vollen Umfang unsere Aufgaben abdeckt.
\nsecend
\nsecbegin{Mögliche Weiterentwicklung}
Denkanstöße:
\begin{itemize}
%\item Wie können wir unser Produkt verbessern, sobald es die definierten Kernfunktionalitäten enthält?
%\item (Am besten mit der Person erörtern, die zu Konkurrenzprodukten recherchiert) Welche Features bietet die Konkurrenz, die wir nicht für die Kernfunktionalität festgelegt haben?
%\item Grob geschätzt: Wie nützlich sind diese Features und wie aufwendig wäre deren Implementierung?
	\item Einlesen von Projekten aus verschiedenen Programmiersprachen / Beispielsweise:
	\begin{itemize}
		\item Beispielsweise:
		\begin{itemize}
			\item Python
			\item C++
			\item C#
			\item etc.
		\end{itemize}
			\item Nützlichkeit:
		\begin{itemize}
			\item Erhöht Einsetzbarkeit des Tools enorm
			\item Erhöht die Anzahl der möglichen Nutzer des Tools
		\end{itemize}
			\item Umsetzbarkeit:
		\begin{itemize}
			\item Durch die XML-Schnittstelle muss alleinig der Parser angepasst werden
			\item Den Parser ist allerdings die aufwendigste Klasse des Programms
		\end{itemize}
	\end{itemize}
	\item Erstellung eines Eclipse Plugins
	\begin{itemize}
		\item Nützlichkeit:
		\begin{itemize}
			%Hier Nutzen
		\end{itemize}
			\item Umsetzbarkeit:
		\begin{itemize}
			%Hier Umsetzen
		\end{itemize}
	\end{itemize}
	
\end{itemize}
%Hier wurde z.B. schon eine Implementierung als Eclipse-Plugin und Support für verschiedene Programmiersprachen angesprochen.
\nsecend

    
     



\nsecbegin{GUI-Anforderungen}
Denkanstöße:
\begin{itemize}
\item Wie könnte das User Interface aussehen?

\item Wie können wir für eine gute User Experience sorgen?
\end{itemize}
Auch, wenn die GUI im ersten Sprint nur in Grundzügen entwickelt wird, sollten wir uns früh Gedanken über deren Aussehen und Funktionalität machen. Dazu gehört, wie der Workflow abläuft und wie wir das GUI unserer Anwendung darauf zuschneiden können. Welche großen roten Knöpfe brauchen wir, welche Funktionen dürfen in dreifach verschachtelten Untermenüs versteckt werden?

Eine Möglichkeit fur das Aussehen und Funktionalität wäre folgende:

\includegraphics[]{Bilder/GUI-Beispiel.png}

Die wichtigsten Funktionalitäten verbergen sich im Untermenü unter Menü.
Ein Projekt bzw. eine Datei mit verschiedenen Klassen soll dort unter „Open“ geöffnet werden.
Das Programm erzeugt daraus automatisch die UML, die rechts entweder grafisch, oder in Textform dargestellt wird. (Auswahl durch Tabs)
Links ist die Klassenhierarchie und der Quelltext der jeweiligen Klasse bzw. Datei zu sehen.
Die erstellte UML soll durch den Benutzer per Maus- und Tastatureingaben in der UML selbst verändert werden können. (Beispiel: Verschieben einer Klasse mit der Maus)
Anschließend soll der Anwender die Möglichkeit haben, die UML grafisch als Bilddatei oder in Textform speichern zu können über die Funktionen „Save as“ und „Export“.
\nsecend
\nsecbegin{Datenmodell}
Denkanstöße:
\begin{itemize}
\item Was für Daten verarbeiten wir überhaupt?
\item In welcher Form werden Daten ein- und ausgegeben?
\end{itemize}
Ein- und Ausgabeform der Daten lässt sich weitgehend aus der Projektbeschreibung ableiten. Ergänzend dazu sollte überlegt werden, welche Variablen / Parameter generell für das Produkt von Bedeutung sind und welche Klassen der Java-Standardbibliothek besonders hilfreich sein könnten.
\nsecend
\nsecbegin{Erste User-Story}
Denkanstöße:
\begin{itemize}
\item Stell dir vor, das Produkt ist gerade fertig entwickelt worden und deine Teamleiter wollen ein Plant-UML von deinem Quellcode sehen (PUMLception). Welchen Workflow erwartest du? Was machst du in welcher Reihenfolge?
\item Optional: Besprich dich mit dem / der Recherchierenden für die GUI-Anforderungen. Habt ihr unterschiedliche Vorstellungen vom Workflow und wenn ja, wie unterscheiden sie sich?
\item Wie würde für dich rein intuitiv das Graphical User Interface aussehen? Wenn du das GUI mit nur drei Knöpfen bauen müsstest, welche Funktionen würdest du ihnen zuweisen?
\end{itemize}
\textbf{Workflow:}
\begin{enumerate}
\item Wähle im Programm „Öffnen“ aus und suche im Explorer die Datei
\item Datei wird eingelesen und verarbeitet
\item Ich sehe das ganze UML Diagramm auf der einen Seite und den Quelltext auf der anderen
\item Im Quelltext kann ich einzelne Parts auswählen, wodurch kleinere UML Diagramme erstellt werden
\item Ich kann die Anordnung der dargestellten Elemente verändern
\item Nun besteht die Möglichkeit das Diagramm z.B. als png-Datei zu exportieren
\end{enumerate}
 

\textbf{GUI:}
Ganz oben wären Button zum minimieren, maximieren und schließen. Darunter eine Menüleiste mit
verschiedenen Optionen um Quelltext und UML-Diagramme zu öffnen und speichern.
Am linken Rand ist die Klassenstruktur, wie man sie z.B. in Eclipse hat dargestellt zur Koordination.
Direkt daneben befindet sich ein Feld mit dem Quelltext, damit man sich direkt dort vergewissern
kann, wie Klassen, die im UML-Diagramm dargestellt sind miteinander im Quelltext interagieren.
Auf der linken Seite ist das UML Diagramm dargestellt.
Die drei Knöpfe wären:
\begin{itemize}
\item Öffnen
\item Speichern
\item Diagramm bearbeiten
\end{itemize}



\nsecend
\nsecbegin{Workshop-Bedarfsermittlung}
Denkanstöße:
\begin{itemize}
\item Was für Kompetenzen könnten dem Team für die Entwicklung des Produkts fehlen?
\end{itemize}
Um die Recherche hier zu vereinfachen, wäre es nicht schlecht, wenn die anderen Recherchierenden den Bearbeitenden auf mögliche Wissenslücken aufmerksam machen. Ziel dieser Recherche ist ein allgemeiner Überblick, wo Defizite vorhanden sind. Es ist zu vermuten, dass die anfänglich identifizierten Probleme eher abstrakter Natur sind - konkrete "Baustellen" zeigen sich für gewöhnlich erst in der Entwicklung. Erwartet werden also keine haargenauen Angaben zu Kompetenzen bzw. Inkompetenzen.
\\
Fehlende Kompetenzen für die Entwicklung des Produktes:
\begin{itemize}
\item Keine UML-Kenntnisse
\item Keine Programmierkenntnisse welche über die bisherigen Anforderungen des Studiums hinausgehen
\item Keine Erfahrungen mit größeren Gruppenarbeiten
\item Keine bisherige GUI-Programmierung mit Hilfe von Tools
\item Noch nie mit GIT gearbeitet
\item Keine Kenntnisse der Fähigkeiten der Gruppenmitglieder
\end{itemize}

Genannte Stichpunkte treffen zwar meist nicht auf alle, aber dennoch auf einen Großteil der Gruppe zu.
Im Laufe des Projekts werden die meisten fehlenden Kompetenzen automatisch wegfallen, da diese auf Erfahrungen aufbauen. Das Arbeiten mit GIT zum Beispiel wird sicherlich nach mehreren Benutzen einfacher.
\nsecend

\nsecbegin{Workshop-Recherche}
Denkanstöße:
\begin{itemize}
\item Wie können wir die Teammitglieder effektiv und effizient auf einen Stand bringen?
\item Welche Ressourcen könnten dafür nützlich sein (Scripting-APIs, gute Tutorials etc.)?
\item Wie können wir die gefundenen Ressourcen so zur Verfügung stellen, dass jeder einfach darauf zugreifen kann?
\end{itemize}
Diese Recherche ist einerseits eng mit der Recherche "Workshop-Bedarfsermittlung" verbunden, es schadet also auf jeden Fall nicht, sich über deren Ergebnisse zu informieren. Aber auch unabhängig davon können schon erste Ideen entwickelt werden, wie das Team miteinander und voneinander lernen kann.
\nsecend



\nsecend

% Das hier ist ein Absatz, der die Grafik in Abbildung~\ref{fig:bild1} detailliert erläutert, erklärt und interpretiert.

% \begin{figure}[b]
%   \centering
%   \includegraphics[width=4.5cm]{bspbild1.png}
%   \caption{Beispiel für ein einspaltiges Bild}
%   \label{fig:bild1}
% \end{figure}


\nsecbegin{Liste der funktionalen Anforderungen}
\nsecbegin{Userstories}
\nsecbegin{Vorschau}
Als Benutzer wünsche ich mir eine Vorschau der Diagramme, damit ich einschätzen kann ob ich damit zufrieden bin.
\nsecend

\nsecbegin{Interfaces}
Als Benutzer wünsche ich mir, dass ich abhängig von Interfaces die zugehörigen Klassen anzeigen lassen kann, damit ich weiß, welche Methoden ich implementieren muss.
\nsecend

\nsecbegin{Kommandozeile}
Als Benutzer wünsche ich mir, dass das Programm von der Kommandozeile aus aufrufbar ist, um es automatisiert starten zu können.
\nsecend

\nsecbegin{Dateien einlesen}
\nsecbegin{Art der eingelesenen Datei}
Als Benutzer wünsche ich mir, dass eine Auswahl zwischen Jar- und Java-Dateien möglich ist, damit Quellcode nicht doppelt eingelesen wird.
\nsecend

\nsecbegin{Java-Dateien}
Als Benutzer wünsche ich mir, dass Java-Dateien einlesbar sind, um den Quellcode von einer oder mehreren Klassen zu analysieren.
\nsecend

\nsecbegin{Jar-Dateien}
Als Benutzer wünsche ich mir, dass Jar-Dateien einlesbar sind, um den Quellcode zu analysieren.
\nsecend
\nsecend

\nsecbegin{Klassendiagramme}
Als Benutzer wünsche ich mir, Klassendiagramme aus meinem bestehenden Quellcode erstellen zu können, damit ich das nicht manuell tun muss.
\nsecend

\nsecbegin{Sequenzdiagramme}
Als Benutzer wünsche ich mir, Sequenzdiagramme aus meinem bestehenden Quellcode erstellen zu können, damit ich das nicht manuell tun muss.
\nsecend

\nsecbegin{Klassenauswahl}
Als Benutzer wünsche ich mir die Möglichkeit, Klassen zu selektieren, damit die Diagramme nicht zu unübersichtlich werden.
\nsecend

\nsecbegin{Methoden- und Variablenauswahl}
Als Benutzer wünsche ich mir die Möglichkeit, Methoden und Variablen zu selektieren, damit die Diagramme nicht zu unübersichtlich werden.
\nsecend

\nsecbegin{Multiple Klassenauswahl}
Als Benutzer wünsche ich mir einen Button, mit dem ich alle Klassen an- oder abwählen kann, damit ich nicht alle Klassen einzeln auswählen muss.
\nsecend

\nsecbegin{Multiple Methodenauswahl}
Als Benutzer wünsche ich mir einen Button, mit dem ich alle Methoden an- oder abwählen kann, damit ich nicht alle Methoden einzeln auswählen muss.
\nsecend

\nsecbegin{Layout}
Als Benutzer wünsche ich mir, das Layout meiner Diagramme ändern zu können, um deren Aussehen zu verbessern.
\nsecend


\nsecbegin{Drag-and-Drop}
Als Benutzer wünsche ich mir, die ausgewählten Dateien oder Ordner per Drag-and-Drop in das Programm aufzunehmen, damit ich den Datei-öffnen-Dialog nicht nutzen muss.
\nsecend

\nsecbegin{Anzeigen und Speichern von PlantUML}
Als Benutzer wünsche ich mir, Diagramme als PlantUML-Code anzeigen und speichern zu können, um den Aufbau nachvollziehen zu können.
\nsecend

\nsecbegin{Speichern von Bilddateien}
Als Benutzer wünsche ich mir, die erstellten Diagramme als Bilddatei exportieren zu können, um sie in meine Projektdokumentation mit aufzunehmen.
\nsecend

\nsecbegin{Speichern von Konfigurationen}
Als Benutzer wünsche ich mir, meine Konfiguration speichern zu können, damit ich meine Präferenz nicht jedes Mal aufs Neue einstellen muss.
\nsecend

\nsecbegin{Benutzerhandbuch}
Als Benutzer wünsche ich mir, das Benutzerhandbuch über die GUI anzeigen lassen zu können, damit ich die gedruckte Version nicht benötige.
\nsecend

\nsecbegin{Übersicht aller Diagramme für ein Projekt}
Als Benutzer wünsche ich mir eine Übersicht aller Diagramme, die ich für mein Projekt erstellt habe, damit ich leichter auf diese zugreifen kann.
\nsecend

\nsecbegin{Drucken}
Als Benutzer wünsche ich mir, Diagramme über das GUI drucken zu können, damit ich die Bilddateien nicht separat öffnen muss.
\nsecend

\nsecbegin{Exceptions als Sequenzdiagramme}
Als Benutzer wünsche ich mir, dass der mögliche Pfad der Exceptions als Sequenzdiagramm angezeigt werden kann, um ungehandelte Exceptions zu vermeiden.
\nsecend

\nsecend
\nsecend

%
% soll der Inhalt dieser Subsection in einer separaten Datei
% (z.B. listefunktional.tex) liegen, dann kann dies mit dem
% folgenden Kommando geschehen.
%
% \input{listefunktional}

\nsecbegin{Liste der nicht-funktionalen Anforderungen}
\nsecbegin{Allgemein}
\begin{itemize}
\nsecend %{Allgemein\textsl{•}}
\item Testabdeckung >= 50\%
\item Keine spürbare Verzögerungen im Programmablauf
\item Bei längeren Ladezeiten muss dies dem Benutzer mitgeteilt werden (Ladebalken)
\end{itemize} 

\nsecbegin{Userstories}
\nsecbegin{Plattformunabhängigkeit}
Als Project Owner wünsche ich mir, dass das Programm plattformunabhängig ist, damit es sich gut verbreiten lässt.
\nsecend
\nsecbegin{Plugin}
Als Benutzer wünsche ich mir, PUML als Plugin direkt in Eclipse verwenden zu können, damit ich nicht außerhalb meiner Entwicklungsumgebung arbeiten muss.
\nsecend
\nsecbegin{Installation}
Als Benutzer wünsche ich mir, dass die Installation unkompliziert ist, damit ich das Programm schnell benutzen kann.
\nsecend
\nsecend %{Userstories}
\nsecend %{Liste der nicht-funktionalen Anforderungen}

\nsecbegin{Weitere Zuarbeiten zum Produktvisions-Workshop}
XXX
\nsecend

\nsecbegin{Zuarbeit von Autor X}
XXX
\nsecend

\nsecbegin{Zuarbeit von Autor Y}
XXX
\nsecend

\nsecbegin{Risikoanalyse}


\begin{table}[H]
\centering
\begin{tabular}{|l|l|l|l|p{8cm}|}
\toprule\addlinespace[2.5cm]
\tabrotate{Warscheinlichkeit} & \tabrotate{Auswirkung} & \tabrotate{Gesamt} & Risiko & Maßnahmen \\
\midrule
\midrule
5 & 5 & 25 & Entwickler fällt aus & \tabitem Die Aufgaben werden umverteilt\\ 
&&&& \tabitem Projektleitung springt ein\\
\midrule
3 & 7 & 21 & Projektleiter fällt aus & Professor Weicker kontaktieren und weitermachen\\
\midrule
7 & 4 & 28 & Die Zeit in einem Sprint reicht nicht & \tabitem Krisentreffen\\
&&&& \tabitem Unterstützung der Verantwortlichen durch andere Entwickler\\
&&&& \tabitem Sprintziel als nicht erreicht kennzeichnen und in nächsten Sprint übernehmen\\
%\midrule
% &  &  &  & \\

\bottomrule
\end{tabular}
\end{table}
\nsecend

\nsecbegin{Liste der Kundengespräche mit Ergebnissen}
\begin{table}[H]

\begin{tabularx}{\textwidth}{ |l|X|X| }
  \hline
  \textbf{Datum} & \textbf{Anliegen oder Fragen} & \textbf{Ergebnisse}\\
  \hline
  \multirow{2}{*}{02.11.18} & Wie genau soll das Layout des Diagramms anpassbar sein? & Das Layout soll sowohl manuell als auch automatisch optimiert werden können. \\\cline{2-3}
  & Reicht es für den ersten Sprint, wenn PUML als Kommandozeilenprogramm umgesetzt wird? & Es soll möglichst früh eine grafische Oberfläche entwickelt werden. Deren Funktionsumfang darf zu Beginn ruhig minimal sein. Wichtig ist, dass das Team möglichst früh einen \glqq optischen Erfolg\grqq{} zu verzeichnen hat. \\
  \hline
\end{tabularx}
\end{table}

\nsecend

\nsecend %Anforderungsspezifikation

\nsecbegin{Architektur und Entwurf}

\nsecbegin{Zuarbeiten der Teammitglieder}
\nsecbegin{Commandline Funktionalität}
Eine der Anforderungen an die Software ist die Bedienung des Programms mittels Kommandozeile. Folgende zwei Möglichkeiten scheinen für die Verwendung im Projekt PUML als sinnvoll. Zum einen ist die Parameterabfrage über eine eigene Implementation auf Basis der Hauptklasse möglich mittels \texttt{public static void main(String [] args)}, zum Anderen ist die Nutzung der \glqq Commons CLI\grqq-Bibliothek von Apache eine Option. \\
Während der Recherche zeigte sich, dass die Nutzung der Commons CLI - Bibliothek sehr gut dokumentiert ist und in der Praxis oft Anwendung findet, auch das Umsetzen eines Tests schien weniger problematisch zu gelingen, als ein Abfragen der Parameter über \texttt{String [] args} im Hauptprogramm. Aus diesem Grund wird an dieser Stelle die Apache Bibliothek kurz vorgestellt. \\
Ein Download der Bibliothek erfolgt über die Seite des Enticklers Apache\footnote[1]{http://commons.apache.org/proper/commons-cli/} und muss anschließend in die Entwicklungsumgebung eingebunden, sowie in das Programm importiert werden. \\
Die Arbeit mit Commons CLI lässt sich grundsätzlich in drei Schritte unterteilen, Parameterdefinition, das Einrichten des Parsers und die Verkettung mit der jeweiligen Funktion. \\
Zuerst wird festgelegt, welche Parameter der Anwendung übergeben werden, hierzu wird ein neues Container Objekt vom Typ \texttt{Options} angelegt. Anschließend werden die gewünschten Befehle mit den entsprechenden Parametern dem Container hinzugefügt, so dass später ein Aufruf im Terminal möglich ist, wie bespielsweise \texttt{ls -al meinfile.txt} um die Zugriffsrechte einer Datei zu überprüfen.
\begin{lstlisting}
//Erzeugt neuen Container fuer Programmparameter
Options options = new Options();
//Hinzufuegen einer neuen Option 
options.addOption("l",false, "Alle Leerzeichen entfernen.");
\end{lstlisting}
Zunächst wird ein Parser intitialisert, während anschließend über eine logische Verknüpfung der Flags die entsprechende Funktion aufgerufen wird. Wichtig ist in diesem Zusammenhang noch die Verwendung von Exceptions zu erwähnen, die entweder durch den Ausdruck \texttt{ParseException} aus der Bibliothek oder \texttt{try / catch} Schlüsselwörter abgefangen werden müssen.
\begin{lstlisting}
CommandLineParser parser = new DefaultParser();	
CommandLine commandLine = parser.parse(options,args);

 if(commandLine.hasOption("b"))
{
	System.out.println("String eingebe: ");
	String myString = keyboard.nextLine();	//String einlesen
	getWordBefore('.',myString); // liefere alle Woerter vor Punkt
}
if (commandLine.hasOption("l"))
{
	String myString = "g e s p e r r t g e s c h r i e b e n";
	System.out.println(noSpace(myString));	//entfernt alle Leerzeichen
}
\end{lstlisting}
Am Ende muss das Programm übersetzt werden und steht anschließend zur Nutzung mit den gesetzten Parametern zur Verfügung. So liefert in diesem Fall die Eingabe im Terminal: \texttt{java MeinProgrammName -l } die Ausgabe \glqq gesperrtgeschrieben\grqq zurück.
\nsecend

\nsecbegin{GUI (Autor: Patrick Otte)}
%Autor: Patrick Otte
\nsecbegin{Welche GUI-Frameworks gibt es für Java?\\[12pt]}
Aktuell gibt es folgende Möglichkeiten zur Erstellung einer GUI Oberfläche in Java:
 \begin{itemize}
	\item Swing
	\item JavaFX
	\item Standard Widget Toolkit (SWT)
	\item Abstract Window Toolkit (AWT)
	\item Google Web Toolkit (GWT)
	\item Qt (Qt Jambi)
	\item GTK+
\end{itemize}
\nsecend %{Welche GUI-Frameworks gibt es für Java?}

\nsecbegin{Welche können für das Projekt genutzt werden?\\}
Für unser Projekt kommen aktuell die Frameworks Swing und JavaFX in Frage. Zum Einen sind beide aktuell die beiden meist genutzten GUI Frameworks in Java, zum Anderen sind hierfür Eclipse Plugins verfügbar.
\nsecend %{Welche können für das Projekt genutzt werden?}

\nsecbegin{Vor- und Nachteile der Frameworks}

\nsecbegin{Swing\\}

\textbf{Vorteile}
\begin{itemize}
	\item Bestandteil des Java Development Kits/Java Foundation Classes
	\item Nutzt eine Sammlung von Bibliotheken zur GUI-Programmierung (Bspw. AWT)
\end{itemize}


\textbf{Nachteile}
\begin{itemize}
	\item Wird nicht mehr weiterentwickelt oder gewartet
	\item Probleme im Bereich Medieneinbindung und Animation
	\item Bestimmte Anwendung wie bspw. Zooming nicht möglich
\end{itemize}

Swing ist eines der meistgenutzten GUI-Frameworks für Java, war bis 2014 ein Standard-Tool zur GUI Entwicklung und hat aufgrund dessen eine große Community hinter sich und man findet viel Hilfestellungen für Swing im Internet.

\nsecend %{Swing}

\nsecbegin{JavaFX\\}

\textbf{Vorteile}
\begin{itemize}
	\item Teil jeder neuen Java SE Installation
	\item Möglichkeit einfach animierte Übergänge einzubinden
	\item optisch ansprechender
	\item Anwendung kann mittels CSS bearbeitet werden (durch Einbindung von FXML-Code)
\end{itemize}


\textbf{Nachteile}
\begin{itemize}
	\item weniger Online-Hilfe, kleinere Community
\end{itemize}


Aufgrund der erst kurzen Zeit, in der JavaFX zur Verfügung steht, gibt es hier viel weniger Hilfe online im Vergleich zu Swing. JavaFX gilt allerdings als der neue Standard in der Java GUI Entwicklung und es gibt sehr viele Developer, die von Swing zu JavaFX umsteigen.

\nsecend %{JavaFX}

\nsecend %{Vor- und Nachteile der Frameworks}

\nsecbegin{Fazit}

In Hinsicht auf die Langlebigkeit bzw. der Zukunftssicherheit, der moderneren Optik, der Einbindung in Eclipse und dem steigenden Support haben wir uns für JavaFX zur GUI Entwicklung in unserem Projekt entschieden.

\nsecend %{Fazit}
\nsecbegin{Update}

Die damalige Entscheidung JavaFX zu verwenden, wurde dann später verworfen. Das damalige (Zusatz)Ziel war es, ein Eclipse-Plugin zusätzlich zu erstellen. Daraufhin kam die Frage auf ob JavaFX mit der IDE kompatibel ist. Dies führte zum Verwerfen von JavaFX als GUI Framework. Im weiteren Verlauf wurde mit SWT gearbeitet.
\nsecend %{Update}
\nsecend %{GUI}%Autor: Patrick Otte
\nsecbegin{JUnit (Autor: Leo Rauschke)}

JUnit ist ein Framework zum Testen von Java-Programmen, das besonders für automatisierte Unit-Tests einzelner Units (Klassen oder Methoden) geeignet ist.Die aktuelle Version ist JUnit 5, bestehend aus JUnit Platform, JUnitJupiter und JUnit Vintage.

In JUnit Jupiter werden die Tests geschrieben, in JUnit Platform ausgeführt und mit JUnit Vintage können ältere Versionen von JUnit - Junit 3 und JUnit4 - durchgeführt werden.

JUnit muss für die Nutzung in Eclipse nicht extra installiert werden, aber die JUnit 5 Library muss in den Build Path des jeweiligen Projektes aufgenommen werden. Von Eclipse gibt es auch eine Anleitung dazu 
%https://www.eclipse.org/eclipse/news/4.7.1a/#junit-5-support
\footnote{\url{https://www.eclipse.org/eclipse/news/4.7.1a/\#junit-5-support}}

Zuerst einmal wird das Projekt erstellt mit einem zusätzlichen Source-Ordner für den Test:
 %Bild1
\begin{figure}[H]
\centering
\includegraphics{Bilder/TRJU_1.png}
%\includegraphics[scale=•]{•}
\end{figure}
 
Mit Rechtsklick auf die Klasse, die getestet werden soll, kann ihr ein JUnit Test Case zugewiesen werden:
 %Bild2
 \begin{figure}[H]
\centering
\includegraphics[width=\textwidth]{Bilder/TRJU_2.png}
%\includegraphics[scale=•]{•}
%\caption{Mit Rechtsklick auf die Klasse, die getestet werden soll, kann ihr ein JUnit Test Case zugewiesen werden}
\end{figure}

 Als Source-Folder wird dabei der dafür angelegte gewählt:
 %Bild3
 \begin{figure}[H]
\centering
\includegraphics{Bilder/TRJU_3.png}
%\caption{Als Source-Folder wird dabei der dafür angelegte gewählt}
%\includegraphics[scale=•]{•}
\end{figure}

Anschließend klickt man auf die Schaltfläche Next > und kann auswählen, was getestet werden soll:  
%Bild4
\begin{figure}[H]
\centering
\includegraphics{Bilder/TRJU_4.png}
%\caption{Anschließend klickt man auf die Schaltfläche Next > und kann auswählen, was getestet werden soll}
%\includegraphics[scale=•]{•}
\end{figure}

Nach Klicken auf den Button Finish muss noch die JUnit 5 library zum Build Path hinzugefügt werden:
 %Bild5
 \begin{figure}[H]
\centering
\includegraphics{Bilder/TRJU_5.png}
%\caption{Nach Klicken auf den Button Finish muss noch die JUnit 5 library zum Build Path hinzugefügt werden}
%\includegraphics[scale=•]{•}
\end{figure}
 Der automatisch erstellte Test kann jetzt  an das eigene Projekt angepasst werden. 
 In einem Test werden die Parameter (hier int a), das erwartete und die Berechnung des tatsächlichen Ergebnisses angegeben. Mit assertEquals(expected, actual); werden die beiden Ergebnisse miteinander verglichen. 
%Bild6
\begin{figure}[H]
\centering
\includegraphics{Bilder/TRJU_6.png}
%\caption{Der automatisch erstellte Test kann jetzt  an das eigene Projekt angepasst werden.  In einem Test werden die Parameter (hier int a), das erwartete und die Berechnung des tatsächlichen Ergebnisses angegeben. Mit assertEquals(expected, actual); werden die beiden Ergebnisse miteinander verglichen. }
%\includegraphics[scale=•]{•}
\end{figure}

Mit dieser Anweisung kann festgelegt werden, dass eine Funktion vor jedem einzelnen Test durchgeführt werden. 
@BeforeEach
public void SetUp() throws Exception {
    classunderTest = new Berechnung();
}

Weitere Anweisungen wie diese sind zum Beispiel @BeforeAll (= bevor irgendein Test durchgeführt wird), @AfterEach (nach jedem Test)und  @AfterAll (nach allen Tests).

Nach Durchführung eines Testes erscheint links anstelle des Package Explorer ein JUnit Tab mit dem Ergebnis des Tests. 
 %Bild7
 \begin{figure}[H]
\centering
\includegraphics{Bilder/TRJU_7.png}
%\includegraphics[scale=•]{•}
%\caption{Nach Durchführung eines Testes erscheint links anstelle des Package Explorer ein JUnit Tab mit dem Ergebnis des Tests.Bei einem nicht bestandenen Test ist der Balken rot und unten wird klein angezeigt, warum der Test nicht bestanden wurde. Mit den drei Buttons rechts über dem Text kann man die Ansicht ändern (vlnr): Konsolenansicht, Filter Stack Trace (default), Side by Side comparison. Bei bestandenem Test ist der Balken grün.}
\end{figure}
Bei einem nicht bestandenen Test ist der Balken rot und unten wird klein angezeigt, warum der Test nicht bestanden wurde.

Mit den drei Buttons rechts über dem Text kann man die Ansicht ändern (vlnr): 
%Konsolenansicht, Filter Stack Trace (default), Side by Side comparison.

Bei bestandenem Test ist der Balken grün.

Ein etwas ausführlicheres Tutorial (auf Englisch) gibt es hier
\footnote{\url{https://www.youtube.com/watch?v=QNv_AQk6WWI}}
\nsecend %{JUnit}
\nsecbegin{Reguläre Ausdrücke}
Reguläre Ausdrücke sind Beschreibungen eines Musters, sog. Patterns, die bei Zeichenkettenverarbeitung eingesetzt werden. Mittels dieser Muster lassen sich Zeichenketten suchen und ersetzen.\\
\nsecbegin{Funktionen}
\begin{itemize}
\item[(1)] Komplette Übereinstimmung suchen \\
\begin{tabular}{l c}
\begin{lstlisting}
Pattern.matches(regex, this);
\end{lstlisting} & \begin{lstlisting}
Pattern p = Pattern.compile(regex);
Matcher m = p.matcher(input);
return m.matches();
\end{lstlisting} \\
\end{tabular}
\item[(2)] Teilstring finden
\begin{itemize}
\item[$\bullet$] alle Vorkommen des Teilstrings innerhalb eines Suchstrings suchen
\end{itemize}
\item[(3)] Teilfolgen ersetzen
\item[(4)] Zerlegen einer Zeichenfolge
\begin{itemize}
\item[$\bullet$] Trennzeichen sind durch Muster definiert, resultiert in Sammlung von Zeichenfolgen
\end{itemize}
\end{itemize}
\nsecend
\nsecbegin{Verwendung}
\begin{itemize}
\item[]Um mit Regulären Ausdrücken arbeiten zu können, wird das Paket ‚java.util.regex‘ implementiert. Es enthält die Klassen Matcher (Zugriff auf Mustermaschine) und Pattern (Repräsentation RE in vorkompiliertem Format).
Außerdem gibt es verschieden Klassifizierungen, um die Suche genauer zu definieren.
\end{itemize}
%Tabelle1
\begin{table} [H]
\centering
\begin{tabular}{l|c}
\multicolumn{1}{l}{\textbf{Quantifizierung}} & \textbf{Anzahl der Wiederholungen}\\
\hline
X? & X kommt einmal oder keinmal vor \\
\hline
X* & X kommt keinmal oder beliebig oft vor \\
\hline
X+ & X kommt einmal oder beliebig oft vor \\
\hline
X\{n\} & X muss genau n-mal vorkommen \\
\hline
X\{n,\} & X kommt mindestens n-mal vor \\
\hline
X\{n,m\} & X kommt mindestens n-, aber max. m-mal vor \\
\hline
\end{tabular}
\end{table}
%Tabelle2
\begin{table} [H]
\centering
\begin{tabular}{l|c}
\multicolumn{1}{l}{\textbf{zeichenklasse}} & \textbf{Enthält}\\
\hline
. & jedes Zeichen \\
\hline
[aei] & Zeichen a, e, i \\
\hline
[\^{}aei] & nicht die Zeichen a, e, i \\
\hline
[0-9a-f] & Zeichen 0-9 oder Kleinbuchstaben a-f \\
\hline
\textbackslash d & Ziffer: [0-9] \\
\hline
\textbackslash D & keine Ziffer: [\^{}0-9] bzw. [\^{}\textbackslash d] \\
\hline
\textbackslash p\{Blank\} & Leerzeichen oder Tab: [\textbackslash t] \\
\hline
\textbackslash p\{Lower\}, \textbackslash p \{Upper\} & Klein-/Großbuchstaben: [a-z] bzw. [A-Z] \\
\hline
\end{tabular}
\end{table}
weitere Klassifizierungen: \\
\href{https://docs.oracle.com/javase/7/docs/api/java/util/regex/Pattern.html}{https://docs.oracle.com/javase/7/docs/api/java/util/regex/Pattern.html}
\nsecend
\nsecbegin{Beispiele}
\begin{itemize}
\item[(1)] Rückgabewert beider Abfragen ist true
\begin{lstlisting}
System.out.println(Pattern.matches("'.*'","'Hallo Welt'" ));
System.out.println("'Hallo Welt'".matches("'.*'"));
\end{lstlisting}
\item[(2)] Abfrage nach Teilstring, Rückgabewert ist gefundener Teilstring
\begin{lstlisting}
String text = "Moderne Programmiersprachen haben durch die Vernetzung von Computern neue Anforderungen erfahren. So lautet auch ein Motto von Sun: 'The Network is the Computer.' ";
Matcher matcher = Pattern.compile("'.*'").matcher(text);
while(matcher.find()) {
	System.out.println(matcher.group());
}
\end{lstlisting}
\end{itemize}
\nsecend
\nsecend
\nsecbegin{XML}
XML ist eine Metasprache die genutzt wird um Daten zwischen Anwendungen auszutauschen. Dies wird durch eine hierarchische Struktur realisiert.\\
\nsecbegin{Eigenschaften}
\begin{itemize}
\item im Format einer Textdatei
\item von Menschen als auch von Maschinen lesbar
\item HTML ist eine Untersprache von XML
\item Dokumenttypdefinitionen (DTD) ermöglichen, dass nur bestimmte Strukturen in einem XML-Dokument möglich sind
\item ohne DTD gut geeignet für beliebigen Datenaustausch
\end{itemize}
\nsecend
\nsecbegin{Vergleich zu JSON (JavaScript Object Notation)}
\begin{itemize} 	

	\item Vorteile gegenüber JSON:
	\begin{itemize}
		\item Einfache Lesbarkeit
		\item Etabliertes Austauschformat
		\item Erweiterbar
	\end{itemize}
	\item Nachteile gegenüber JSON:
	\begin{itemize}
		\item Enthält viel "Balast" der für Datenaustausch nicht nötig ist
		\item Datenvolumen relativ hoch
		\item Komplexe Syntax 
	\end{itemize}		
\end{itemize}
\begin{figure}
	\centering
	\includegraphics[scale=0.6]{Bilder/XML-Beispiel}
	\caption{Beispielcode XML}
	\label{fig:bild2}
\end{figure}
\nsecend
\nsecbegin{Verwendung}
\begin{itemize}
	\item Erste Zeile im Dokument definiert Version und Codierung
\item Struktur in der Form <tag> ... </tag>, wobei <x> das öffnende Tag und </x> das schließende Tag darstellt
\item Ein Wurzelknoten wird benötigt, der den gesamten XML-Quelltext umfasst
\item Tags sind ineinander geschachtelt
\item Attribute können mit Attribut =''Wert'' im öffnenden Tag definiert werden		
\end{itemize}
\nsecend
\nsecbegin{Bibliotheken in Java}
\begin{itemize}
\item org.xml.sax.* (XML Datei lesen)
\item org.w3c.dom.* (einlesen in den Speicher und schreiben in der Datei)
\item java.xml.parsers.* (Auslesen der XML-Dateien aus dem Speicher und übernehmbar als DOM-Objekt)
\end{itemize}
\nsecend
\nsecend
\nsecbegin{XPath (Autor: Michael Lux)}
Abfragesprache zur Addressierung/Auswertung von XML und Grundlage für Standards: XLST, XPointer, XQuery.
\nsecbegin{Aufbau und Verwendung}
\begin{itemize} 
	
	\item XML-Dokument wird als Baum betrachtet
	\begin{itemize}
		\item Knoten (nodes): Dokumenten-Knoten, XML-Elemente, -Attribute, -Textknoten, -Kommentare, -Namensräume und -Verarbeitungsanweisungen
		\item Achsen: preceding, following, preceding-sibling und following-sibling	
	\end{itemize}

\end{itemize}

\begin{itemize} 
	
	\item XPath-Ausdruck besteht aus mehreren Lokalisierungsschritten:
	\begin{itemize}
		\item achse::knotentest[prädikat 1][prädikat 2]...
		\item Beispiel: /descendant-or-self::Foo
		\item Prädikate: Funktionen/Operatoren zur weiteren Einschränkung
		\item Beispiel: text(), comment() für bestimmten Datentyp	
	\end{itemize}

\end{itemize}
\nsecend
\nsecbegin{Beispiel an dargestellter XML-Datei}
\begin{itemize}
	\item /descendant-or-self::Softwareprojekt bzw. Softwareprojekt
	
	Wählt alle untergeordnete Knoten inklusive des Kontextknotens Softwareprojekt aus.
	
	\item /descendant-or-self::Softwareprojekt/descendant::Gruppe bzw. Softwareprojekt//Gruppe
	
	Wählt alle untergeordnete Knoten von Gruppe aus.
	
	\item /descendant-or-self::Softwareprojekt/descendant::Gruppe/descendant::Master/descendant::Projektowner/attribute::* oder /Softwareprojekt/Gruppe/Master/Projektowner/attribute::*
	
	liefert Attribute='mID=1' zurück.
	
\end{itemize}
\nsecend
\nsecend
%Autor: Jan Sollmann

%\UseRawInputEncoding
\nsecbegin{Softwareverbreitung unter Windows (Autor: Jan Sollmann)}
	\begin{itemize}
	\item Beschreibung
	
	
	Um eine unbeaufsichtigte Installation von Software zu gewährleisten benötigt unser Projekt für Windows ein installierbares Softwarepaket. Um ein solches Softwarepaket zu erstellen, benötigt man die Hilfe eines Installers.
	
	\item Installer-Typen
	
	
    \begin{itemize}
	\item Kriterien
	
	
	Kriterien bezüglich der Installer sind zum einen, ob selbige mit Open-Source arbeiten. Zudem sollten die Installer im Bezug auf das nicht-kommerzielle Projekt nicht kostenpflichtig sein. Im Generellen sollte der ideale Installer außerdem eine einfache Handhabung innehaben und im Spezifischen die Überprüfung, ob die JRE installiert ist, und gegebenenfalls die Installation derselbigen anbieten.
	
	\item Nullsoft Scriptable Install System (NSIS)
	
	
	\textbf{NSIS} bietet ein kostenlosen aber sehr flexiblen wie auch minimalen Installer. Jedoch sind durch Open-Source bereits viele Plugins vorhanden. Auch hier ist eine Überprüfung und Installation der JRE möglich. Der Compiler der NSIS's ist jedoch recht primitiv gestaltet und ein intuitives Verständnis des Codes gestaltet sich durch fehlendes Syntax-Highlighting schwer.
	
	\item Inno Setup
	
	
	Das \textbf{Inno Setup} ist ein Open-Source-unterstützender Installer welcher kostenlos downloadbar ist. Der Installer kommt mit einem übersichtlichen Compiler, einfacher und gut strukturierter Syntax, wie einer Code-Sektion in welcher komplexe Vorgänge mit Pascal programmiert werden können, und zudem mit vorgefertigten Beispielen zu bestimmten Software-Typen. Auch eine Überprüfung und eventuelle Installation der JRE ist umsetzbar. 
	\end{itemize}
	\item Fazit
	
	
	Nach Betrachtung beider Installer genügen beide Typen vollkommen den gestellten Anforderungen. Jedoch ist das \textit{Inno Setup} durch übersichtlichen Compiler und intuitiver Handhabung meine persönliche Empfehlung im Sinne des Softwareprojekts.
	\end{itemize}

\nsecend



\nsecbegin{Exceptions}
Mit Exceptions reagiert man auf Fehler und unerwartete Situationen während der Ausführung eines Programms. Diese werden mit einem try-Block erstellt und mit einem catch-Block abgefangen. Es existieren viele Möglichkeiten wodurch Exception auftreten können. Möglich sind beispielsweise überschrittene Arraygrenzen, Zugriff auf nicht erzeugte Objekte oder fehlerhafte Typkonvertierungen. Mit dem Code throw new Exception(); lässt sich bewusst eine Exception erstellen.\\
In den Java-Bibliotheken gibt es bereits viele Exceptiontypen. Allerdings ist es auch möglich eigene als Unterklasse der Exceptionklasse zu erstellen, wenn die vorhandenen nicht ausreichen.\\\\
Benötigte Exceptions für unser Projekt: \\
Für die Konsolenanwendung wären Exceptions sinnvoll. Mit diesen könnte man beispielsweise fehlerhafte Pfadangaben abfangen. Da in der GUI die Dateien über den Explorer eingelesen werden, sollten dort keine falschen Pfade zustande kommen.\\
Beispiel für Exceptions:
\begin{lstlisting}
public class Main
{
	public static void main(String[] args)
	{		
		int[] numberArray = { 1, 2, 0 };
		try
		{
			//System.out.println(1 / numberArray[2]);
			System.out.println(numberArray[3]);
		}
		catch (ArithmeticException exception)
		{
			System.out.println("Nicht durch Null teilen!");
		}
		catch (Exception e)
		{
			System.out.println("Fehler: " + e);
		}
		System.out.println("Programm wird trotz Fehler weiterhin ausgeführt");
	}
}
\end{lstlisting}
Da das Programm auf eine Stelle im numberArry zugreifen möchte, welche nicht existiert, kommt es zu der Fehlermeldung. Kommentiert man die zweite anstelle der ersten Systemausgabe aus, so kommt es zu einer anderen Fehlermeldung: „Nicht durch Null teilen!“ Das Beispiel zeigt, dass durch das Catchen verschiedener Fehlertypen verschiedene Befehle ausgeführt werden können.
\nsecend

\nsecend

\nsecbegin{Entscheidungen des Technologieworkshops}
\begin{itemize}
\item Als GUI-Framework wird SWT verwendet
\item JUnit-Tests werden in Eclipse ausgeführt
\item Es soll XML als Datenformat verwendet werden
\end{itemize}
\nsecend

\nsecbegin{Überblick über Architektur}
\begin{figure}[H]
\centering
\includegraphics[width=\textwidth]{Bilder/classDiagrammFinal.png}
\caption{Klassendiagramm des Projekts}
\end{figure}
\nsecend

\nsecbegin{Definierte Schnittstellen}
\begin{figure}[H]
\centering
\includegraphics[scale=0.6]{Bilder/Schnittstellen.png}
\caption{Schnittstellen des Projekts}
\end{figure}
\nsecend

\nsecbegin{Liste der Architekturentscheidungen}
%XXX (bewusste und unbewusste Entscheidungen mit zeitlicher Einordnung)

\begin{table}[H]
\centering
\begin{tabular}{|l|l|}
\toprule
Zeit & Entscheidung \\
\midrule
Bei Projektvergabe & Als Programmiersprache wird Java verwendet.\\
& Begründung der Entscheidung:\\
&      \tabitem Plattformunabhängigkeit\\
&      \tabitem Alle aus dem Team beherrschen Java\\
&      \tabitem Sauber und einsteigerfreundlich\\
&      \tabitem Weiterentwicklung zu Eclipse-Plugin möglich\\
&      \tabitem Ausgereifte GUI-Frameworks verfügbar\\
\midrule
Technologieworkshop & Für die GUI wird SWT verwendet\\
& Begründung der Entscheidung:\\
&      \tabitem Einfach zu verwenden\\
&	   \tabitem Ausgereift\\
&	   \tabitem Umfangreich\\
&	   \tabitem Editor als Eclispse-Plugin\\
\midrule
Beginn Sprint 3 & Umbau von ParsingResult zu XML\\
& Begründung der Entscheidung:\\
&	   \tabitem Schon länger geplant aber bisher zu komplex\\
&	   \tabitem Da die Grundfunktionalität besteht, Umbau jetzt möglich\\
&	   \tabitem Klarer modularer Aufbau möglich\\
&	   \tabitem Bessere Verteilung der Aufgaben durch Modularität\\
&	   \tabitem Zwischenschritte können gegen Spezifikation getestet werden\\
\midrule
Beginn Sprint 3 & Wechsel von SWT zu AWT/SWING\\
& Begründung der Entscheidung:\\
&	   \tabitem Trotz vieler versuche läuft SWT-Tool nicht sauber unter Linux\\
&	   \tabitem Das Einbinden der SWT-Bibliotheken verursacht Probleme beim \\ &  wechsel zwischen Windows und Linux\\
&	   \tabitem AWT/SWING ist sehr ausgereift\\
&	   \tabitem Es müssen keine zusätzlichen Bibliotheken eingebunden werden\\
&	   \tabitem Tool für die Entwicklung läuft unter Windows und Linux \\ & zuverlässig als Eclipse-Plugin\\
\bottomrule
\end{tabular}
\end{table}
\nsecend


\nsecend %Architektur und Entwurf

\end{shownto} %{-, developer, manualDE}
\begin{shownto}{-, manualDE}

\nsecbegin{Prozess- und Implementationsvorgaben}

\nsecbegin{Definition of Done}
XXX
\nsecend

\nsecbegin{Coding Style}
Bitte die Datei javaCodeStyle.xml im specification-Verzeichniss in Eclipse importieren und verwenden.
Hierfür in Eclipse unter "`Window->Preferences->Java->Code Style->Formatter"' auf Import klicken und die XML-Datei auswählen.\\
Ist der passende Coding Style eingestellt kann der Quellcode mit "`STRG+SHIFT+F"' automatisch formatiert werden.
Wird dies vor jedem Commit gemacht, ensteht ein einheitlicher Code-Style und die Änderungen können gut mit GIT überprüft werden.\\
\begin{figure}[hbtp]
\centering
\includegraphics[scale=0.5]{Bilder/importCodeStyle}
\caption{Code-Style in Eclipse importieren}
\end{figure}
Des weiteren empfiehlt es sich bei größeren oder stark geschachtelten Code-Abschnitten die Zugehörigkeit der Schließenden Klammer mit einem Kommentar zu Kennzeichnen.\\
Sonstige Konventionen:
\begin{itemize}
\item{Variablen und Instanzen beginnen kleingeschrieben}
\item{Klassen und Interfaces beginnen mit Großbuchstaben}
\item{Besteht ein Namen aus mehreren zusammengesetzten Wörtern, beginnen alle weiteren Wörter mit Großbuchstaben (keine Unterstriche in Namen verwenden)}
\item{Aussagekräftige Namen verwenden}
\item{Alle Namen auf Englisch}
\item{Die Kommentare auf Deutsch}
\item{Lange Kommentare immer vor den Codeabschnitt}
\item{Alle Methoden im Javadoc-Stiel dokumentieren}
\end{itemize}

\nsecend

\nsecbegin{Zu nutzende Werkzeuge}
\begin{itemize}
\item{Eclipse - Entwicklungsumgebung}
\item{GIT - Dateiversionierung}
\item{Meld - Unterschiede zwischen Dateien anzeigen}
\item{Texmaker - Latex-Editor}
\item{GIMP - Bildbearbeitung für das Editieren von Screenshots}
\end{itemize}
\nsecend



\nsecend %Prozess- und Implementationsvorgaben

\end{shownto} %{-, manualDE}
\begin{shownto}{developer}
\nsecbegin{Was wird wie gemacht?}
\nsecbegin{Eclipse}
\nsecbegin{Projekt in Eclipse importieren}
In den workspace wecheln:\\
cd ~/workspace\\
Projekt Klonen:\\
git clone https://gitlab.imn.htwk-leipzig.de/weicker/puml.git\\
Benutzername und Passwort eingeben.\\
In Eclipse "File->Import->Existing Projects into Workspace"\\
%\begin{figure}[hbtp]
%\centering
\includegraphics[scale=0.25]{Bilder/importProject}\\
%\caption{Projekt in Eclipse importieren}
%\end{figure}
Dann auf "'Finish"' klicken.
\nsecend

\nsecbegin{WindowBuilder installieren}
In Eclipse "Help->Install New Software..."\\
Unter work with "'2018-09 - http://download.eclipse.org/releases/2018-09"' auswählen.\\
In der Section "'General Purpose Tools"' die im Bild stehenden Häckchen anklicken\\	
\begin{figure}[hbtp]
\centering
\includegraphics[scale=0.4]{Bilder/installWindowBuilder}\\
\caption{WindowBuilder installieren}
\end{figure}
Dann auf "'Finish"' und sich durch die Installation klicken.
\nsecend

\nsecbegin{GUI editieren}
Es muss der WindowBuilder installiert sein. Dann auf die Datei die die Grafische Oberfläche implementiert (GUI.java) mit der rechten Maustaste klicken. Dann "'Open With->WindowBuilder Editor"' auswählen.
\nsecend
\nsecend

\nsecend
\nsecbegin{Best practice}
\nsecbegin{GIT}
\nsecbegin{Keine nicht benötigten Dateien adden}
Vor dem "'git add ."' immer mit "'git status"' prüfen welche Dateien hinzugefügt werden. Sollten nicht für das Projekt benötigte Dateien (z.B. übersetzte Binärdateien oder Dokumentation von Librarys) dabei sein, bitte die entsprechende "'.gitignore-Datei"' vervollständigen. Danach sollten die Dateien beim "'git status"' nicht mehr angezeigt und somit nicht mehr geadded werden.\\
HINWEIS: Die "'.gitignore-Datei"' ist (wie an dem führenden Punkt zu sehen ist) versteckt und wird nur nach dem setzen des entsprechenden Häckchens im Dateimanager oder beim "'ls -a"' angezeigt.
\nsecend
\nsecend
\nsecend
\end{shownto}

\begin{shownto}{-, developer, manualDE}

%%%%%%%%%%%%
%% Abschnitt mit den Sprints beginnt hier
%%%%%%%%%%%%

\nsecbegin{Sprint 1}
%Author: Tore Arndt, Patrick Otte
\nsecbegin{Ziel des Sprints}
Es soll eine funktionsfähige Basisversion, welche für das einfache erstellen von Klassendiagrammen aus Java-Code verwendet werden soll entstehen. Das Programm soll sowohl über die Kommandozeile, als auch über eine grafische Oberfläche bedient werden können. Die erzeugten Klassendiagramme sollen in der grafischen Oberfläche angezeigt werden können.

\begin{figure}[hbtp]
\centering
\includegraphics[scale=0.5]{Bilder/classDiagrammSprint1}
\caption{Klassendiagramm des Sprints}
\end{figure}
\nsecend

\nsecbegin{User-Stories des Sprint-Backlogs}
\nsecbegin{Dateien einlesen}
\nsecbegin{Art der eingelesenen Datei}
Als Benutzer wünsche ich mir, dass eine Auswahl zwischen Jar- und Java-Dateien möglich ist, damit Quellcode nicht doppelt eingelesen wird.
\nsecend

\nsecbegin{Java-Dateien}
Als Benutzer wünsche ich mir, dass Java-Dateien einlesbar sind, um den Quellcode von einer oder mehreren Klassen zu analysieren.
\nsecend

\nsecbegin{Jar-Dateien}
Als Benutzer wünsche ich mir, dass Jar-Dateien einlesbar sind, um den Quellcode zu analysieren.
\nsecend
\nsecend

\nsecbegin{Vorschau}
Als Benutzer wünsche ich mir eine Vorschau der Diagramme, damit ich einschätzen kann ob ich damit zufrieden bin.
\nsecend

\nsecbegin{Kommandozeile}
Als Benutzer wünsche ich mir, dass das Programm von der Kommandozeile aus aufrufbar ist, um es automatisiert starten zu können.
\nsecend

\nsecbegin{Klassendiagramme}
Als Benutzer wünsche ich mir, Klassendiagramme aus meinem bestehenden Quellcode erstellen zu können, damit ich das nicht manuell tun muss.
\nsecend

\nsecbegin{Anzeigen und Speichern von PlantUML}
Als Benutzer wünsche ich mir, Diagramme als PlantUML-Code anzeigen und speichern zu können, um den Aufbau nachvollziehen zu können.
\nsecend

\nsecbegin{Plattformunabhängigkeit}
Als Project Owner wünsche ich mir, dass das Programm plattformunabhängig ist, damit es sich gut verbreiten lässt.
\nsecend
\nsecend % {User-Stories des Sprint-Backlogs}

\nsecbegin{Zeitliche Planung}
\begin{figure}[hbtp]
\centering
\includegraphics[width=\textwidth]{Bilder/gantt}
\caption{Gantt-Diagramm für Sprint 1}
\end{figure}
\nsecend

\nsecbegin{Liste der durchgeführten Meetings}
\begin{itemize}
\item Planning-Meeting (29.11.2018)
\item Zwischen-Meeting (03.12.2018)
\item Review-Meeting (13.12.2018)
\end{itemize}
\nsecend

\nsecbegin{Ergebnisse des Planning-Meetings}
Dem gesammten Team ist die geplante Grundstruktur des Programms bekannt. Jeder weiß welchen Teil des Programms er implementieren soll.
\nsecend

\nsecbegin{Aufgewendete Arbeitszeit pro Person$+$Arbeitspaket}
\begin{longtable}{|p{4cm}|l|l|l|l|l|}
        \hline
        Arbeitspaket & Person & Start & Ende & h & Artefakt\\
        \hline
        PUML-26/36 & Patrick Otte & 26.11.18 & 21.12.18 & 14 & OutputPUML.java\\ \hline
        PUML-26/37 & Tore Arndt & 26.11.18 & 21.12.18 & 14 & OutputPUML.java\\ \hline
        PUML-24/29 & Leo Rauschke & 26.11.18 & 21.12.18 & 11,5 & CodeCollector.java\\ \hline  
        PUML-24/30 & Elisabeth Schuster & 26.11.18 & 21.12.18 & 11 & CodeCollector.java\\ \hline  
        PUML-25/31 & Jona Meyer & 26.11.18 & 21.12.18 & 30 & Parser.java\\ \hline  
        PUML-24/32 & Michael Lux & 26.11.18 & 21.12.18 & 17 & Parser.java\\ \hline  
        PUML-27/34 & Johann Gerhardt & 26.11.18 & 21.12.18 & 10 & Console.java\\ \hline
        PUML-27/35 & Marian Geißler & 26.11.18 & 21.12.18 & 11 & Console.java\\ \hline
        PUML-28/33 & Jan Sollmann & 26.11.18 & 21.12.18 & 12 & GUI.java\\ \hline
        PUML-28/38 & Julian Uebe & 26.11.18 & 21.12.18 & 14 & GUI.java\\ \hline
\end{longtable}     
\nsecend

\nsecbegin{Konkrete Code-Qualität im Sprint}
Es gibt an vielen Stellen noch erheblichen Optimierungsbedarf. Teils Code doppelt anstatt in Methoden oder durch bessere Struktur nur einmal vorhanden. Des Weiteren fehlen an vielen Stellen Kommentare und Dokumentationen zu den jeweiligen Methoden. Die vorher vom Softwarearchitekten vorgegebenen Coding Conventions und Styles müssen umgesetzt und beachtet werden, sodass, bei einem später geplanten eigenen Merge, Konflikte vermieden werden können.
\nsecend

\nsecbegin{Konkrete Test-Überdeckung im Sprint}
Testüberdeckung liegt bei 41,2\%.
\nsecend

\nsecbegin{Ergebnisse des Reviews}
\begin{table}[H]

\begin{tabularx}{\textwidth}{ |l|l|X| }
\hline
\textbf{Klasse} & \textbf{Methode} & \textbf{Anmerkungen}\\
 \hline
Console & showConsole & Pfad anpassen \\
CodeCollector & - & Unit-Tests für Ordner \\
CodeCollector & getSourceCode & gleichzeitig .jar- und .java-Dateien \\
ParserJava & parse & Bug: Entfernt zu viel Source Code! Mehr Tests\\
OutputPuml & - & generell mehr Kommentare \\
OutputPuml & getPuml & Redundanter Code mit savePumlToFile, generell mehr Kommentare\\
OutputPuml & createPlantUML & Performance verbessern \\
GUI\_SWT & - & Entwicklerdokumentation (Installationsanleitung) für verwendetes Tool\\
\hline
\end{tabularx}
\end{table}

Sonstiges:
\begin{itemize}
\item mehr Kommentare
\item (Graphviz muss installiert sein, um PlantUML anzuzeigen)
\item Javadocs schreiben!
\item in gitconfig Name und Mail-Adresse anpassen! Wichtig für Benotung!
\item Ordner für Unit-Tests ist srcTest
\item im Ordner srcTest ein Unterordner "'testfiles"' ertellen, in dem zusätzliche Testdateien landen
\end{itemize}
\nsecend

\nsecbegin{Ergebnisse der Retrospektive}
\begin{figure}[hbtp]
\centering
\includegraphics[scale=0.4]{Bilder/Retrospektive_Tafelbild.jpg} 
\caption{Anmerkungen der Teammitglieder zum bisherigen Verlauf des Projekts}
\end{figure}
Die Retrospektive schloss, sowohl was den Lernerfolg als auch die Kommunikation im Team angeht, mit einer positiven Bilanz. Gelobt wurde besonders das Gruppenklima und die Zusammenarbeit sowie die Verteilung der fertig eingerichteten Virtual Machine. Kollektive Lernerfolge sind besonders in den Bereichen Git und Latex zu verzeichnen, daneben decken die genannten spezialisierten Bereiche die Themenfelder ab, die den jeweiligen Gruppen zugeteilt wurden. Für die Zukunft hofft das Team auf günstigere Zeiten für die Projekttreffen. Bemängelt wurde, dass es bei der Einführung in das Arbeiten mit Git allgemein Defizite gab. Spezifisch machte besonders die Bedienung des Meld-Tools Schwierigkeiten. Es wurde deshalb beschlossen, auf diese Probleme in einem der folgenden Gruppentreffen noch einmal ausführlich einzugehen.\\
Befürchtet wurde vor allen Dingen, dass im nächsten Semester mehr Fehler auftreten werden, als das Team zunächst vermutet hätte. Um das Risiko zu verringern, dass sich diese Befürchtungen bewahrheiten, wurden bereits bekannte Bugs zusammengetragen und in Jira gestellt. Ziel des zweiten Sprints ist die vollständige Implementierung aller für den ersten Sprint definierten Use Cases (sofern noch nicht erfolgt) wie auch das Beheben aller bisher dokumentierten Bugs. Die Praxis des Pair Programmings wird zunächst beibehalten. Angestrebt ist dennoch eine bessere Dokumentierung des Quellcodes, auch, um eine potenzielle Umverteilung der Teammitglieder zu erleichtern. Im nächsten Semester soll für die Teammitglieder die Möglichkeit bestehen, auf Wunsch in einen anderen Teilbereich des Projektes \glqq hineinzuschnuppern\grqq. Auch die in Jira angelegten Issues sollen besser getrennt werden, um sie auch an Einzelpersonen zuweisen zu können.\\ \\
\nsecend

\nsecbegin{Abschließende Einschätzung des Product-Owners}
Insgesamt konnten die meisten für den ersten Sprint definierten Use Cases implementiert werden. Damit existiert bereits eine minimale, lauffähige Version des Programms. Werden im nächsten Sprint die noch nicht vollständig im ersten Sprint implementierten Funktionen fertiggestellt sowie enthaltene Bugs entfernt, ist eine solide Grundlage für die weitere Entwicklung des Produkts gelegt.
\nsecend

\nsecbegin{Abschließende Einschätzung des Software-Architekten}
Die elementarsten Funktionalitäten sind implementiert. Die für die Architektur entworfenen Schnittstellen greifen wie geplant ineinander. 
\nsecend

\nsecbegin{Abschließende Einschätzung des Team-Managers}
Für dieses Projektteam ist die Rolle des Team-Managers nicht vergeben. Von den Ergebnissen der Retrospektive ausgehend lässt sich allerdings annehmen, dass die bisherige Vorgehensweise bei der Organisation des Projekts sowie der Durchführung der Treffen und des ersten Sprints grundsätzlich ein guter Ansatz zu sein scheint. 
\nsecend


\nsecend

\nsecbegin{Sprint 2}
\nsecbegin{Ziel des Sprints}
Dieser Sprint ist ausschließlich dazu gedacht, im Verlauf des ersten Sprints identifizierte und noch nicht behobene Bugs zu entfernen. Es wurden bewusst keine neuen User-Stories (für den Benutzer) im Sprint-Backlog definiert. Der Fokus liegt darauf, die im ersten Sprint geschaffene Basis noch einmal zu stabilisieren.
\nsecend

\nsecbegin{User-Stories des Sprint-Backlogs}
\nsecbegin{Reduzierung von Bugs}
Als Softwarearchitekt und Product Owner wünschen wir uns, dass möglichst wenige Bugs auftreten, um die spätere Weiterentwicklung und damit die uneingeschränkte Funktionalität des Produkts nicht zu gefährden.
\nsecend
\nsecend % {User-Stories des Sprint-Backlogs}

\nsecbegin{Zeitliche Planung}
\iffalse
\begin{figure}[hbtp]
\centering
\includegraphics[width=\textwidth]{Bilder/diagram_sprint2}
\caption{Kontroll-Diagramm für Sprint 2}
\end{figure}
\fi
\nsecend%Zeitliche Planung

\nsecbegin{Liste der durchgeführten Meetings}
\begin{itemize}
\item Planning-Meeting (21.02.2019)
\item Zwischen-Meeting (08.04.2019)
\item Zwischen-Meeting (11.04.2019)
\item Review-Meeting (15.04.2019)
\end{itemize}
\nsecend%Liste der durchgeführten Meetings

\nsecbegin{Ergebnisse des Planning-Meetings}
Der zweite Sprint wird zeitlich in der letzten Woche der Semesterferien begonnen und bis zum Ende der ersten Woche der Vorlesungszeit gehen. Dies wurde mit den Teammitgliedern besprochen. Hauptziel des Sprints ist ein sauberer Stand, mit dem ab dem kommenden Sommersemester weitergearbeitet werden kann.
\nsecend

\nsecbegin{Aufgewendete Arbeitszeit pro Person$+$Arbeitspaket}
\begin{longtable}{|p{4cm}|l|l|l|l|l|}
        \hline
        Arbeitspaket & Person & Start & Ende & h & Artefakt\\
        \hline
        Testdaten & Marian Geissler   & 06.04.2019 & 15.04.2019 & 2 &  \\ \hline
        Logger & Patrick Otte   & 08.02.2019 & 08.04.2019 & 12 & LogMain.java \\ \hline
        Output & Patrick Otte   & 11.04.2019 & 15.04.2019 & 3 & OutputPUML.java \\ \hline
        Konsole & Johann Gerhardt   & 14.04.2019 & 14.04.2019 & 1 & Console.java \\ \hline
        Java-Parser & Michael Lux   & 30.03.2019 & 30.03.2019 & 14 & ParserJava.java\\ \hline
        GUI & Jan Sollmann  & 01.04.2019 & 06.04.2019 & 5 & GUI\_SWT.java \\ \hline
        GUI & Julian Uebe  & 21.02.2019 & 15.04.2019 & 7 & SWT-Tool \\ \hline
        Code-Collector & Elisabeth Schuster  & 07.02.2019 & 14.04.2019 & 5.5  & CodeCollector.java \\ \hline
        Profiler & Elisabeth Schuster  & 10.04.2019 & 26.04.2019 & 7  & Profiler \\ \hline
       Java-Parser & Jona Meyer  & 30.03.2019 & 30.03.2019 & 7 & ParserJave.java \\ \hline
        Code-Collector & Leo Rauschke  & 07.04.2019 & 14.04.2019 & 5.75 & CodeCollector.java \\ \hline
        Profiler & Leo Rauschke  & 09.04.2019 & 29.04.2019 & 3 & Profiler\\ \hline
        Output & Tore Arndt  & 11.04.2019 & 15.04.2019 & 5 & OutputPUML.java\\ \hline
        
        
\end{longtable}     
\nsecend

\nsecbegin{Konkrete Code-Qualität im Sprint}
Die Code-Qualität wurde durch den Fokus auf die Behebung bestehender Bugs verbessert.
\nsecend%Konkrete Code-Qualität im Sprint

\nsecbegin{Konkrete Test-Überdeckung im Sprint}
Die Test-Überdeckung in diesem Sprint war überdurchschnittlich.
\nsecend%Konkrete Test-Überdeckung im Sprint

\nsecbegin{Ergebnisse des Reviews}
\begin{table}[H]

\begin{tabularx}{\textwidth}{ |l|l|X| }
\hline
\textbf{Klasse} & \textbf{Methode} & \textbf{Anmerkungen}\\
 \hline
 
 Testdatensatz & komplett & zukünftig als automatischer Ausgabe-Test\\ \hline
 GUI\_SWT.java & komplett & wird durch Swing-GUI ersetzt\\ \hline
 CodeCollector & Pfadbehandlung & Funktioniert nun auch unter Windows\\ \hline
 CodeCollector & einlesen & Funktioniert\\ \hline
 ParserJava.java & buildTree & Es bestehen weitherhin Bugs\\ \hline
 Alle & komplett & Unit-Tests für das ganze Programm folgen\\ \hline
 ParserJava.java & buildTree & Erweiterung für das Erstellen von Sequenzdiagrammen\\ \hline
 GUI\_SWT.java & createContents, runPUML & Ausgabe für Sequenzdiagramme muss implementiert werden\\ \hline
 
%Console & showConsole & Pfad anpassen \\
\hline
\end{tabularx}
\end{table}

\nsecend%Ergebnisse des Reviews

\nsecbegin{Ergebnisse der Retrospektive}
Die Retrospektive schloss mit einer positiven Bilanz. (...)
\nsecend%Ergebnisse der Retrospektive

\nsecbegin{Abschließende Einschätzung des Product-Owners}
In diesem Sprint wurden einige kritische Bugs behoben und somit User Stories des letzten Sprint-Backlogs noch vervollständigt. Zu hoffen bleibt trotzdem, dass in allen folgenden Sprints auch neue Funktionalität hinzugefügt wird.
\nsecend%Abschließende Einschätzung des Product-Owners

\nsecbegin{Abschließende Einschätzung des Software-Architekten}
XXX
\nsecend%Abschließende Einschätzung des Software-Architekten

\nsecbegin{Abschließende Einschätzung des Team-Managers}
Insgesamt kann positiv herausgehoben werden, dass sich die Teammitglieder geschlossen dazu bereit erklärten, Teil ihrer Semesterferien für den zweiten Sprint zu opfern. Die Motivation, das Produkt weiter voranzubringen, scheint derzeit ungebrochen.
\nsecend%Abschließende Einschätzung des Team-Managers
\nsecend

\nsecbegin{Sprint 3}
\nsecbegin{Ziel des Sprints}
Die bestehenden Funktionen des Programms sollen um die Möglichkeit ergänzt werden, neben Klassendiagrammen auch Sequenzdiagramme erstellen zu können. Dafür muss die Struktur der verarbeiteten Daten, also auch der Code zum Parsen, angepasst werden. Geplant ist eine Repräsentation des eingelesenen Codes als zentrale XML-Datei, die je nach Anwendungszweck wiederum die Basis für zwei unterschiedliche XML-Dateien ist.


% HIER NEUES KLASSENDIAGRAMM HINZUFUEGEN
\begin{figure}[hbtp]
\centering
\includegraphics[width=\textwidth]{Bilder/classDiagrammSprint3.png}
\caption{Klassendiagramm des Sprints}
\end{figure}




\nsecend %{Ziel des Sprints}

\nsecbegin{User-Stories des Sprint-Backlogs}
\nsecbegin{Sequenzdiagramme}
\nsecbegin{Auswahl des zu erstellenden Diagramms}
Als Benutzer wünsche ich mir, dass eine Auswahl zwischen Klassen- und Sequenzdiagrammen möglich ist, damit ich diese je nach meinen Bedürfnissen generieren kann.
\nsecend

\nsecbegin{Generierung von Sequenzdiagrammen}
Als Benutzer wünsche ich mir, Sequenzdiagramme erstellen zu können, um einen Überblick über die Abläufe meines Programms zu erhalten.
\nsecend
\nsecend%Sequenzdiagramme

\nsecbegin{Interne Struktur}
\nsecbegin{Parserfunktionalität}
Als Softwarearchitekt wünsche ich mir, dass der Parser effizient funktioniert, um die benötigten Daten ohne Schwierigkeiten auszulesen.
\nsecend

\nsecbegin{Dateiformat XML}
Als Softwarearchitekt wünsche ich mir, dass der übergebene Code zur weiteren Verarbeitung in verschiedene XML-Dateien umgewandelt wird.
\nsecend
\nsecend%Interne Struktur
\nsecend % {User-Stories des Sprint-Backlogs}

\nsecbegin{Kontroll-Diagramm}
\begin{figure}[hbtp]
\centering
\includegraphics[width=\textwidth]{Bilder/KD-Jira.png}
\caption{Kontroll-Diagramm für Sprint 3}
\end{figure}
\nsecend%Zeitliche Planung

\nsecbegin{Liste der durchgeführten Meetings}
\begin{itemize}
\item Planning-Meeting (15.04.2019)
\item Parser-Besprechung (18.04.2019)
\item Parser-Besprechung (24.04.2019)
\item Zwischen-Meeting (25.04.2019)
\item Review-Meeting (29.4.2019)
\end{itemize}
\nsecend%Liste der durchgeführten Meetings

\nsecbegin{Ergebnisse des Planning-Meetings}
Dem gesamten Team ist die geplante Grundstruktur des Programms bekannt. Jeder weiß, welcher Teil des Programms zu implementieren ist.
\nsecend
\nsecbegin{Aufgewendete Arbeitszeit pro Person$+$Arbeitspaket}
\begin{longtable}{|p{4cm}|l|l|l|l|l|}
        \hline
        Arbeitspaket & Person & Start & Ende & h & Artefakt\\
        \hline
   
        Profiler in Betrieb nehmen & Elisabeth Schuster & 26.4.19 & 26.4.19 & 1h &  \\ \hline
        
        
        BugfixParser & Michael Lux & 15.4.19 & 15.4.19 & 2h & ParserJava.java\\ \hline
        
        BugfixParser & Jona Meyer & 28.4.19 & 29.4.19 & 10h & ParserJava.java\\ \hline
       
      

        Unit-tests für den Gesammtaufruf & Marian Geißler & 17.4.19 & 28.4.19 & 7h & MainTest.java\\ \hline


        Generator für Sequenzdiagramme - Spezifikation für xml-Datei & Leo Rauschke & 22.4.19 & 22.4.19 & 1h & SeqDiagram.xml\\ \hline
        
        Generator für Sequenzdiagramme & Leo Rauschke & 18.4.19 & 18.4.19 & 3h 15m & SequenceDiagramGenerator.java \\ \hline      
       Generator für Sequenzdiagramme & Elisabeth Schuster & 22.4.29 &  24.4.19 & 3h 30m & SequenceDiagramGenerator.java\\ \hline\\       
       Übergabe der Klassendiagramme als XML & Tore Arndt & 21.4.19  & 24.4.19 & 25h & OutputPUML.java\\ \hline
        Übergabe der Klassendiagramme als XML & Patrick Otte & 21.4.19  & 24.4.19 & 30h 15m & OutputPUML.java\\ \hline   
       GUI zu AWT/SWING umbauen & Julian Uebe & 15.4.19 & 16.4.19 & 8h & GUI_Swing.java \\ \hline\\
       
        
       
\end{longtable}     
\nsecend

\nsecbegin{Konkrete Code-Qualität im Sprint}
Die Code-Qualität ist bisher gleichmäßig. Da die Umstrukturierung des Parsers besondere Sorgfalt erfordert, wurden zusätzliche Treffen mit den verantwortlichen Teammitgliedern angesetzt.
\nsecend%Konkrete Code-Qualität im Sprint

\nsecbegin{Konkrete Test-Überdeckung im Sprint}
Die Test-Überdeckung in diesem Sprint fiel eher niedrig aus, da für die neuen Funktionen erst passende Testdateien erstellt bzw. bestehende Testdateien angepasst werden müssen.
\nsecend%Konkrete Test-Überdeckung im Sprint

\nsecbegin{Ergebnisse des Reviews}
\begin{table}[H]

\begin{tabularx}{\textwidth}{ |l|l|X| }
\hline
\textbf{Klasse} & \textbf{Methode} & \textbf{Anmerkungen}\\
 \hline
%Console & showConsole & Pfad anpassen \\
\hline
\end{tabularx}
\end{table}

Sonstiges:
\begin{itemize}
\item Zum Testen des überarbeiteten Parsers müssen die Testdateien angepasst werden
\end{itemize}
\nsecend%Ergebnisse des Reviews

\nsecbegin{Ergebnisse der Retrospektive}
Die Retrospektive schloss mit einer gemischten Bilanz. Besonders die Umstrukturierung des Parsers warf viele Fragen auf, da nicht nur der Code des Parsers selbst, sondern auch alle Schnittstellen angepasst werden mussten. Hier zeichnete sich eine leichte Unzufriedenheit ab, da Teile der bisher geleisteten Arbeit nun umgeschrieben werden. Für den nächsten Sprint ist es daher wünschenswert, die Erzeugung der neuen XML-Dateien sowie darauf aufbauend der Sequenzdiagramme voranzutreiben, damit hier möglichst bald ein Erfolg zu verzeichnen ist.
\nsecend%Ergebnisse der Retrospektive

\nsecbegin{Abschließende Einschätzung des Product-Owners}
Insgesamt ist ein beachtlicher Teil der definierten Sprintziele noch in Bearbeitung. Da es sich um einen kurzen Sprint handelte und zudem große Teile des Codes geändert werden müssen, um die Generation von Sequenzdiagrammen vorzubereiten, ist dies nachvollziehbar. 
\nsecend%Abschließende Einschätzung des Product-Owners

\nsecbegin{Abschließende Einschätzung des Software-Architekten}
XXX
\nsecend%Abschließende Einschätzung des Software-Architekten

\nsecbegin{Abschließende Einschätzung des Team-Managers}
Vom allgemeinen Sprintverlauf ausgehend wird das Erstellen von Sequenzdiagrammen die größte Herausforderung des Projekts werden. Damit die betreffenden Teammitglieder im Vergleich nicht zu viel Zeit in den Parser investieren müssen, wurden zwei zusätzliche Treffen mit dem Softwarearchitekten angesetzt, bei denen die notwendigen Voraussetzungen besprochen und die Methodik erörtert wurde.
\nsecend%Abschließende Einschätzung des Team-Managers

\nsecend

\nsecbegin{Sprint 4}
\nsecbegin{Ziel des Sprints}
Ziel des Sprints ist, einen möglichst funktionalen Zwischenstand zu erreichen, der die Spezifikation erfüllt. Insbesondere Unit-Tests für die einzelnen Teile sowie die Weiterentwicklung der Sequenzdiagramm-Generierung sollen implementiert werden.

% HIER NEUES KLASSENDIAGRAMM HINZUFUEGEN
%\begin{figure}[hbtp]
%\centering
%\includegraphics[scale=0.5]{}
%\caption{Klassendiagramm des Sprints}
%\end{figure}
\nsecend

\nsecbegin{User-Stories des Sprint-Backlogs}
\nsecbegin{Sequenzdiagramme}s
\nsecbegin{Auswahl des zu erstellenden Diagramms}
Als Benutzer wünsche ich mir, dass eine Auswahl zwischen Klassen- und Sequenzdiagrammen möglich ist, damit ich diese je nach meinen Bedürfnissen generieren kann.
\nsecend

\nsecbegin{Generierung von Sequenzdiagrammen}
Als Benutzer wünsche ich mir, Sequenzdiagramme erstellen zu können, um einen Überblick über die Abläufe meines Programms zu erhalten.
\nsecend
\nsecend%Sequenzdiagramme
\nsecend % {User-Stories des Sprint-Backlogs}

\nsecbegin{Zeitliche Planung}
\begin{figure}[hbtp]
\centering
\includegraphics[width=\textwidth]{Bilder/gantt}
\caption{Gantt-Diagramm für Sprint 1}
\end{figure}
\nsecend%Zeitliche Planung

\nsecbegin{Liste der durchgeführten Meetings}
\begin{itemize}
\item Planning-Meeting (29.04.2019)
\item Zwischenstandspräsentation (06.05.2019)
\item Review-Meeting (10.5.2019)
\end{itemize}
\nsecend%Liste der durchgeführten Meetings

\nsecbegin{Ergebnisse des Planning-Meetings}
Die XML- und Textspezifikationen und deren Transformationen sind allen Teams bekannt. Die Grundlage für die Weiterentwicklung der Sequenzdiagramm-Generierung ist damit gelegt. 
\nsecend

\nsecbegin{Aufgewendete Arbeitszeit pro Person$+$Arbeitspaket}
\begin{longtable}{|p{4cm}|l|l|l|l|l|}
        \hline
        Arbeitspaket & Person & Start & Ende & h & Artefakt\\
        \hline
        Dummyklassen & Musterstudi & 3.5.09 & 12.5.09 & 14 & Klasse.java\\ \hline
        AP XYZ &  &  &  & & \\ \hline
\end{longtable}     
\nsecend

\nsecbegin{Konkrete Code-Qualität im Sprint}
Aufgrund der Umstrukturierung sind einige Funktionalitäten noch nicht gegeben. Das Durchlaufen der XML-Bäume sollte teilweise noch angepasst werden. Hierfür bietet sich die Verwendung von XPath-Ausdrücken an.
\nsecend%Konkrete Code-Qualität im Sprint

\nsecbegin{Konkrete Test-Überdeckung im Sprint}
Es fehlen noch einige Unit-Tests.
\nsecend%Konkrete Test-Überdeckung im Sprint

\nsecbegin{Ergebnisse des Reviews}
\begin{table}[H]

\begin{tabularx}{\textwidth}{ |l|l|X| }
\hline
\textbf{Klasse} & \textbf{Methode} & \textbf{Anmerkungen}\\
 \hline
%Console & showConsole & Pfad anpassen \\
\hline
\end{tabularx}
\end{table}

Sonstiges:
\begin{itemize}
\item XPath nutzen
\item Rekursive Implementierung beim Abarbeiten von verschachtelten Elementen
\item Fehlende Unit-Tests nachholen
\item Evtl. in späterem Sprint: Handling von Methoden in Bedingungen
\end{itemize}
\nsecend%Ergebnisse des Reviews

\nsecbegin{Abschließende Einschätzung des Product-Owners}

\nsecend%Abschließende Einschätzung des Product-Owners

\nsecbegin{Abschließende Einschätzung des Software-Architekten}
XXX
\nsecend%Abschließende Einschätzung des Software-Architekten

\nsecbegin{Abschließende Einschätzung des Team-Managers}

\nsecend%Abschließende Einschätzung des Team-Managers
\nsecend

\nsecbegin{Sprint 5}
\nsecbegin{Ziel des Sprints}
Ziel des Sprints ist es, die Erzeugung von Klassendiagrammen und Sequenzdiagrammen zu verbessern. Dies soll erreicht werden durch eine Verfeinerung der XML-Strukturspezifizierung der Klassen- und Sequenzdiagramm-Generatoren und durch Erstellung einheitlicher Funktionen zur Traversierung selbiger Struktur, sowie durch Änderungen am Parser selbst. Die Funktionalität des Programms soll durch erweiterte Konsolenbefehle verbessert werden.
\nsecend

\nsecbegin{User-Stories des Sprint-Backlogs}
\nsecbegin{Vorschau der Diagramme}
Als Benutzer wünsche ich mir eine Vorschau der Diagramme, damit ich einschätzen kann ob ich
damit zufrieden bin.
\nsecend
\nsecbegin{Anzeigen und Speichern von PlantUML}
Als Benutzer wünsche ich mir, Diagramme als PlantUML-Code anzeigen und speichern zu können, um den Aufbau nachvollziehen zu können.
\nsecend
\nsecbegin{Erstellung von Klassendiagrammen}
Als Benutzer wünsche ich mir, Klassendiagramme aus meinem bestehenden Quellcode erstellen zu können, damit ich einen Überblick über die Klassen meines Programms und deren Beziehungen bekomme.
\nsecend
\nsecbegin{Generierung von Sequenzdiagrammen}
Als Benutzer wünsche ich mir, Sequenzdiagramme erstellen zu können, um einen Überblick über die Abläufe meines Programms zu erhalten.
\nsecend
\nsecend % {User-Stories des Sprint-Backlogs}

\nsecbegin{Zeitliche Planung}
Der Sprint geht vom 13.05.19 bis zum 24.05.19.
\nsecend%Zeitliche Planung

\nsecbegin{Liste der durchgeführten Meetings}
\begin{itemize}
\item Planning-Meeting (13.05.2019)
\item Zwischen-Meeting (17.05.2019)
\item Review-Meeting (24.05.2019)
\end{itemize}{}
\nsecend%Liste der durchgeführten Meetings

\nsecbegin{Ergebnisse des Planning-Meetings}
Die Zuweisung der Aufgabengebiete fiel auf die Teams zurück, die diese bereits zuvor angegangen sind, da diese mit ihnen vertraut sind und sonst zusätzliche Zeit für die Einarbeitung notwendig wäre.
Es wurde beschlossen, dass die Klassen für die Generierung von Klassen- und Sequenzdiagrammen sowie die Hilfsfunktionen zur Traversierung der XML-Struktur vollständig auf die Verwendung von XPath-Ausdrücken umgestellt werden sollen. 
\nsecend

\nsecbegin{Aufgewendete Arbeitszeit pro Person$+$Arbeitspaket}
\begin{longtable}{|p{4cm}|l|l|l|l|l|}
        \hline
        Arbeitspaket & Person & Start & Ende & h & Artefakt\\
        \hline
        GUI in develop-Branch integrieren & Julian U. & 13.05.2019 & 24.05.2019 & 2 & GUISwing.java \\ \hline
        Generator für Sequenzdiagramme & Leo R. & 13.05.2019 & 24.05.2019 & 19 & SequenceDiagramGenerator.java\\ \hline
        Generator für Sequenzdiagramme & Elisabeth S. & 13.05.2019 & 24.05.2019  & 21 & SequenceDiagramGenerator.java \\ \hline
        Ausgabe für Sequenz-und Klassendiagramme & Tore A.  & 13.05.2019 & 24.05.2019  & 8  & SequenceDiagramGenerator.java \\ \hline
        Ausgabe für Sequenz-und Klassendiagramme & Patrick O.  & 13.05.2019 & 24.05.2019  & 8  & SequenceDiagramGenerator.java \\ \hline
        Generator für Klassendiagramme & Johann G.  & 13.05.2019 & 24.05.2019  & 2 & ClassDiagrammGenerator.java \\ \hline
        Sequenz-Diagramm-Tree und Umbau & Jan S. & 13.05.2019 & 24.05.2019 & 3 & SequenceDiagramGenerator.java \\ \hline
        Alte Codefragmente entfernen  & Marian G. & 13.05.2019 & 24.05.2019 & 2.5 & OutputPUML.java ParserJava.java\\ \hline
        Anpassung Console auf neue Methoden & Marian G. & 13.05.2019 & 24.05.2019  & 6  & Console.java \\ \hline
        Java Parser umbauen & Jona M. & 13.05.2019 & 24.05.2019 & 12 & ParserJava.java \\ \hline
        Java Parser umbauen & Michael L. & 13.05.2019 & 24.05.2019 & 8 & ParserJava.java \\ \hline

\end{longtable}     
\nsecend

\nsecbegin{Konkrete Code-Qualität im Sprint}
Lesbarkeit ist annehmbar aber durchaus noch optimierbar. Immer wieder werden Bugs gefunden und behoben.
\nsecend%Konkrete Code-Qualität im Sprint

\nsecbegin{Konkrete Test-Überdeckung im Sprint}
Die Testüberdeckung ist mit 60,2\% deutlich gestiegen.
\nsecend%Konkrete Test-Überdeckung im Sprint

\nsecbegin{Ergebnisse des Reviews}

\begin{table}[H]

\begin{tabularx}{\textwidth}{ |l|l|X| }
\hline
\textbf{Klasse} & \textbf{Methode} & \textbf{Anmerkungen}\\
 \hline
ClassGeneratorTest.java & allgemein & Testklasse fur Classdiagrammgenerator erstellt, die erzeugtes Diagramm mit Vorlage ClassDiagram.xml vergleicht\\ \hline
SequenzDiagramGenerator.java & allgemein & Funktionen kommentiert, Abfangen von Exceptions  (u.a. XPathExpressionException), alten Code durch XMLHelperMethods wie getChildwithName ersetzt\\ \hline
SequenzDiagramGenerator.java & handleLocalInstances & Funktion eingefügt, die die Klassen-Tags von lokalen Instanzen in das Dokument einfügt\\ \hline
SequenzDiagramGenerator.java & addClassesToInstances & Hinzufügen der Klassen-Tags in methodcalls funktioniert grundlegend (mit kleinen Bugs)\\ \hline
ClassDiagramGenerator.java & createDiagram & Verbesserte Suche nach Vererbungen, Kompositionen, Aggregationen und Implementierungen (funktioniert), alten Code entfernt\\ \hline
GUI\_Swing.java & initialize & GUI auf XML umgestellt und entsprechende Klassen eingebunden, Baumstruktur (JTree) zur Auswahl des Einstiegspunktes überarbeitet\\ \hline
GUI\_Swing.java & initialize & Tree für Sequenzdiagramm erstellt, GUI-Tabulatoren umgestellt\\ \hline
XmlHelperMethods.java & removeComments & Funktion eingefuegt, die Kommentare aus XML-Baum entfernt\\ \hline
XmlHelperMethods.java & removeWhitespace & Funktion eingefügt, die Document als String ohne Leerzeichen zurück gibt\\ \hline
OutputPUML.java & allgemein & Neuimplementation der Erstellung von PUML aus Klassendiagramm mit Verwendung von getList mit Referenzknoten, Umbau der Iteration durch XML Diagramm für Methoddefinitionen\\ \hline
ParserJava.java & buildTree & Grundlegende Struktur für Erkennung von Funktionen angelegt\\ \hline
 
  
 

\hline
\end{tabularx}
\end{table}

Die Ausgabe von Klassendiagrammen funktioniert, es gibt an mehreren Stellen noch Codefehler. Die Grundstruktur wurde ausgearbeitet, es fehlen jedoch noch u.a. die umfangreiche Erkennung von Funktionen.
Beim Review wurden die Änderungen an der XML-Struktur der Klassen- und Sequenzdiagrammen, an den Hilfsfunktionen zur Auswertung selbiger und die neuen Konsolenbefehle vorgestellt, damit alle Teammitglieder mit diesen vertraut sind und mit ihnen arbeiten können.
Für die Änderungen am Parser und an den Klassen- u. Sequenzdiagrammen wurden die Tests erweitert, bzw. neu erstellt. (z.B. ClassDiagram2.xml)

In den Klassen OutputPUML, ParserIF, ParserJava wurde zahlreich nicht mehr benötigter Code entfernt.

\nsecend%Ergebnisse des Reviews

\nsecbegin{Abschließende Einschätzung des Product-Owners}
Mit dem weiter voranschreitenden Umbau werden bald neue Funktionalitäten integriert sowie die bestehenden Funktionalitäten verbessert werden.
\nsecend%Abschließende Einschätzung des Product-Owners

\nsecbegin{Abschließende Einschätzung des Software-Architekten}
Der Umbau ist weitestgehend abgeschlossen. Die neue GUI wurde in den Hauptzweig integriert. Es können wieder Klassendiagramme erzeugt werden, auch wenn diese noch fehlerbehaftet sind.
\nsecend%Abschließende Einschätzung des Software-Architekten
\nsecend

\nsecbegin{Sprint 6}
\nsecbegin{Ziel des Sprints}
Ziel des Sprints ist es unter anderem alte Codefragmente zu entfernen, einige Klassen auszubessern und an neue Vorraussetzungen anzupassen und mit dem Erstellen des C++ Parsers zu beginnen. Vor allem sollte am Ende des Sprints das Programm lauffähig sein.
\nsecend

\nsecbegin{User-Stories des Sprint-Backlogs}
\nsecbegin{Vorschau}
Als Benutzer wünsche ich mir eine Vorschau der Diagramme, damit ich einschätzen kann ob ich
damit zufrieden bin.
\nsecend
\nsecbegin{Klassendiagramme}
Als Benutzer wünsche ich mir, Klassendiagramme aus meinem bestehenden Quellcode erstellen zu können, damit ich das nicht manuell tun muss.
\nsecend
\nsecbegin{Anzeigen und Speichern von PlantUML}
Als Benutzer wünsche ich mir, Diagramme als PlantUML-Code anzeigen und speichern zu können, um den Aufbau nachvollziehen zu können.
\nsecend
\nsecend

\nsecbegin{Zeitliche Planung}
Der Sprint geht vom 24.05.19 bis zum 07.06.19.
\nsecend

\nsecbegin{Liste der durchgeführten Meetings}
\begin{itemize}
\item Planning-Meeting (24.05.2019)
\end{itemize}
\nsecend

\nsecbegin{Ergebnisse des Planning-Meetings}
\begin{itemize}
\item Bugfixes und Feinschliff Sequenzdiagrammgenerator [PUML-92]
\item XML Compare für UnitTests [PUML-93]
\item GUI Auflistung Klassen/Methoden [PUML-94]
\item C++ Parser schreiben [PUML-95]
\item Sequenzdiagramm Output fertigstellen [PUML-96]
\item Test openjdk oracle jdk [PUML-98]
\item alte Codefragmente entfernen [PUML-84]
\item Anpassung Konsole auf neue Methoden [PUML-85]
\item Parser umbauen [PUML-89/90]
\item alten Code entfernen
\item PUML Logo bestimmt
\end{itemize}
\nsecend

\nsecbegin{Aufgewendete Arbeitszeit pro Person$+$Arbeitspaket}
\begin{longtable}{|p{4cm}|l|l|l|l|l|}
        \hline
        Parser umbauen & Michael Lux & 24.05.19 & 07.06.19 & 1w 4d 3h & ParserJava.java\\
        \hline
        Parser umbauen & Jona Meyer & 24.05.19 & 07.06.19 & 4d 6h & ParserJava.java\\
        \hline
        Bugfix und Feinschliff von Sequenzdiagrammen & Elisabeth Schuster & 24.05.19 & 07.06.19 & 1d 6h 15m & SequenzDiagramGenerator.java\\
        \hline
        SeqDiagram1 & Leonie Rauschke & 24.05.19 & 07.06.19 & 5h 50m & SequenzDiagramGenerator.java\\
        \hline 
        XML-Compare für Unit-Tests & Patrick Otte & 24.05.19 & 05.06.19 & 2d & \\
        \hline
        Logger Logdatei & Patrick Otte & 24.05.19 & 03.06.19 & 3h &\\
        \hline
        getList Methode & Patrick Otte & 26.05.19 & 27.05.19 & 6h & \\
        \hline\\
        Anpassung Console auf neue Methoden & Marian Geißler & 24.05.19 & 07.06.19 & 1d 6h 30m & \\
        \hline
        GUI auflistung der Klassen und Methoden & Julian Uebe & 24.05.19 & 07.06.19 & 1w 7h & \\
        \hline
        C++-Parser erstellen & Jan Sollmann & 24.05.19 & 07.06.19 & 1w 3d 1h 30m  & ParserCPP.java\\
        \hline
        C++-Parser erstellen & Johann Gerhardt & 24.05.19 & 07.06.19 & 4d 1h  & ParserCPP.java\\
        \hline
        Sequenzdiagramm-Output fertigstellen & Tore Arndt & 24.05.19 & 07.06.19 & 2d 4h & \\
        \hline   
        Test für Sequenzdiagramm Output schreiben & Patrick Otte & 24.05.19 & 27.05.19 & 4h & \\
        \hline     
\hline
\end{longtable}     
\nsecend

\nsecbegin{Konkrete Code-Qualität im Sprint}
Alte und unbenötigte Codefragmente wurden entfernt. In einigen Klassen wurden Bugs behoben. Die Code-Qualität hat sich daher wahrscheinlich verbessert.
\nsecend%Konkrete Code-Qualität im Sprint

\nsecbegin{Konkrete Test-Überdeckung im Sprint}
Es wurde eine Methode für das Vergleichen von XML-Dokumenten geschrieben. Mit dieser lassen sich beispielsweise der ClassDiagramGenerator und der SequenzDiagramGenerator testen. Daher sind in diesem Sprint neue Tests hinzugekommen, wodurch die Testabdeckung hoch gehalten wird.
Insgesamt ist die Testüberdeckung mit 63,5\% leicht gestiegen und im annehmbaren Bereich.
\nsecend%Konkrete Test-Überdeckung im Sprint

\nsecbegin{Ergebnisse des Reviews}
\begin{table}[H]

\begin{tabularx}{\textwidth}{ |l|l|X| }
\hline
\textbf{Klasse} & \textbf{Methode} & \textbf{Anmerkungen}\\
\hline
\end{tabularx}
\end{table}

\nsecend%Ergebnisse des Reviews

\nsecbegin{Ergebnisse der Retrospektive}
Der Sprint lief insgesamt betrachtet erfolgreich. Allerdings wird der CPP-Parser wohl nicht die gleiche Funktionalität wie der Java-Parser erreichen. Das Parsen von CPP-Quellcode funktioniert zwar in Ansätzen, ist allerdings weder vollständig noch für komplexeren Code ausgelegt.
\nsecend%Ergebnisse der Retrospektive

\nsecbegin{Abschließende Einschätzung des Product-Owners}
Der weiterentwickelte Java-Parser funktioniert bis auf wenige Einschränkungen. Da es sich hierbei um das umfangsreichste Modul handelte, kann dies als wichtiger Zwischenerfolg angesehen werden.
\nsecend%Abschließende Einschätzung des Product-Owners

\nsecbegin{Abschließende Einschätzung des Software-Architekten}
Der Java-Parser nimmt langsam Form an. Da sich die für das Projekt vorhandene Zeit dem Ende naht, wurden folgende Ziele für den Projektabschluss vereinbart:
\begin{itemize}
\item Klassendiagramme sollen aus jedem Java-Code generiert werden können
\item Sequenzdiagramme sollen zumindest für die Java-Spezifikation funktionieren
\item Klassendiagramme sollen auch für die C++-Spezifikation funktionieren
\end{itemize}
\nsecend%Abschließende Einschätzung des Software-Architekten
\nsecend

\nsecbegin{Sprint 7}
\nsecbegin{Ziel des Sprints}
Ziel des Sprints ist, ein funktionaler Zustand der Software. Die Bedienung soll möglichst fehlerfrei über GUI und Kommandozeile ermöglicht werden. Des weiteren sollen alte Codefragmente entfernt und die Struktur der einzelnen Klassen weiterhin verbessert werden. Da ein Ende des Projektes langsam absehbar ist, sollen Unit-Tests für eine größere Testüberdeckung sorgen, die einzelnen Teile sowie die Weiterentwicklung der Sequenzdiagramm-Generierung (Exceptions visualisiert) sollen angepasst werden, jedoch keine weitere Grundlegende Funktionalität hinzugefügt werden. \\
Zusätzlich werden weitere Methoden in der XML-Hilfsklasse implementiert und für bessere Fehlerbehandlung der Logger erweitert. Weitere Auswahloptionen für eine verbesserte Nutzerinteraktion sollen in GUI und Konsole implementiert, sowie kleinere Bugfixes vorgenommen werden.\\

% HIER NEUES KLASSENDIAGRAMM HINZUFUEGEN
%\begin{figure}[hbtp]
%\centering
%\includegraphics[scale=0.5]{}
%\caption{Klassendiagramm des Sprints}
%\end{figure}
\nsecend

\nsecbegin{User-Stories des Sprint-Backlogs}
\nsecbegin{Exceptions als Sequenzdiagramme}
Als Benutzer wünsche ich mir, dass der mögliche Pfad der Exceptions als Sequenzdiagramm angezeigt werden kann, um ungehandelte Exceptions zu vermeiden.
\nsecend
\nsecbegin{Klassendiagramme}
Als Benutzer wünsche ich mir, Klassendiagramme aus meinem bestehenden Quellcode erstellen zu können, damit ich das nicht manuell tun muss.
\nsecend
\nsecbegin{Klassenauswahl}
Als Benutzer wünsche ich mir die Möglichkeit, Klassen zu selektieren, damit die Diagramme nicht zu unübersichtlich werden.
\nsecend
\nsecbegin{Kommandozeile}
Als Benutzer wünsche ich mir, dass das Programm von der Kommandozeile aus aufrufbar ist, um es automatisiert starten zu können.
\nsecend
\nsecbegin{Methoden- und Variablenauswahl}
Als Benutzer wünsche ich mir die Möglichkeit, Methoden und Variablen zu selektieren, damit die Diagramme nicht zu unübersichtlich werden.
\nsecend
\nsecbegin{Multiple Klassenauswahl}
Als Benutzer wünsche ich mir einen Button, mit dem ich alle Klassen an- oder abwählen kann, damit ich nicht alle Klassen einzeln auswählen muss.
\nsecend
\nsecbegin{Sequenzdiagramme}
\nsecbegin{Auswahl des zu erstellenden Diagramms}
Als Benutzer wünsche ich mir, dass eine Auswahl zwischen Klassen- und Sequenzdiagrammen möglich ist, damit ich diese je nach meinen Bedürfnissen generieren kann.
\nsecend

\nsecbegin{Generierung von Sequenzdiagrammen}
Als Benutzer wünsche ich mir, Sequenzdiagramme erstellen zu können, um einen Überblick über die Abläufe meines Programms zu erhalten.
\nsecend
\nsecend%Sequenzdiagramme
\nsecend % {User-Stories des Sprint-Backlogs}

\nsecbegin{Zeitliche Planung}
Für die Zeitliche Planung wurde von der ursprünglichen Idee, den Sprint eine Woche dauern zu lassen abgesehen, da ein Feiertag auf das Datum der Zwischenstandspräsentationen fiel. Somit wurde der Sprint 7 vom 07.06.2019 bis zum 17.06.2019 angesetzt.
%\begin{figure}[hbtp]
%\centering
%\includegraphics[width=\textwidth]{Bilder/gantt}
%\caption{Gantt-Diagramm für Sprint 1}
%\end{figure}
\nsecend%Zeitliche Planung

\nsecbegin{Liste der durchgeführten Meetings}
\begin{itemize}
\item Planning-Meeting (07.07.2019)
\item Meeting zur Präsentation des Zwischenstands (14.06.2019)
\item Review-Meeting (17.06.2019)
\end{itemize}
\nsecend%Liste der durchgeführten Meetings

\nsecbegin{Ergebnisse des Planning-Meetings}
Neben der Vereinbarung des angestrebten Funktionsumfangs, wurden weitere Bedingungen für Teile der Software zusammengetragen. Damit wurde die Spezifikation wie folgt erweitert:\\
\begin {itemize}
\item Variablen der Basisdatentypen eingefügt
\item Zugriffsmodifikatoren für Instanzen, Variablen und Methoden eingefügt
\item Konstruktor ist nun auch eine Methode in parsedData.xml
\item static-modifier sollte nun geparsed und in Klassendiagramme übernommen werden
\item Alle Methoden die kein Konstruktor sind, haben nun einen Result-Wert (ggf. void)
\item Eigene parsedData.xml für c++
\end {itemize}
Außerdem soll die Software nun auch bei komplexeren Klassendiagrammen ein zuverlässiges Ergebnis liefern. Bezüglich der Sequenzdiagramme soll die Grundfunktionalität erreicht werden, verschachtelte Aufrufe, wie bspw. instanz.methode().methode() oder instanz.methode(instanz.methode) sollen vorerst nicht berücksichtigt werden, jedoch nicht zum Absturz des Programms führen.\\
Die Entwicklung des C++ Parsers wird vorerst auf Klassendiagramme eingeschränkt.\\
Das Anzeigen und Auswählen der Klassen fürs Klassendiagramm in GUI und Konsole sollte funktionieren, sowie das Anzeigen und Auswählen der Eintrittsmethode für das Sequenzdiagramm.
Die Klasse ''ClassDiagrammgenerator'' soll über die boolischen Variablen: showInstances, showVars und showMethods verfügen und von außen gesetzt werden können.
Die Grundlage für die Weiterentwicklung und das Erreichen eines soliden Funktionsumfangs der Software wurde damit gelegt.
\nsecend

\nsecbegin{Aufgewendete Arbeitszeit pro Person$+$Arbeitspaket}
\begin{longtable}{|p{4cm}|l|l|l|l|l|}
        \hline
        Arbeitspaket & Person & Start & Ende & h & Artefakt\\
        \hline
        Console & Marian G. & 07.06.2019 & 17.06.2019 & 23 & Console.java\\ \hline
        Sequenzdiagramm & Tore A.  & 07.06.2019 & 17.06.2019  &20 & SequenceDiagramGenerator.java \\ \hline
        Sequenzdiagramm & Patrick O.  & 07.06.2019 & 17.06.2019  & 4  & SequenceDiagramGenerator.java \\ \hline
        Logger & Patrick O.  & 07.06.2019 & 17.06.2019  & 1  & Logger.java \\ \hline
        Hilfsklasse xml & Leo R.  & 07.06.2019 & 17.06.2019  & 4  & XmlHelperMethods.java \\ \hline
        Klassendiagramm & Johann G.  & 07.06.2019 & 17.06.2019  & 29 & ClassDiagrammGenerator.java \\ \hline
        Hilfsklasse xml  & Patrick O.  & 07.06.2019 & 17.06.2019  & 1 & XmlHelperMethods.java \\ \hline
        C++ Parser  & Jan S.  & 07.06.2019 & 17.06.2019  & 8 & XmlHelperMethods.java \\ \hline
        Java Parser  & Jona M.  & 07.06.2019 & 17.06.2019  & 16 & ParserJava.java \\ \hline
        Java Parser  & Michael L.  & 07.06.2019 & 17.06.2019  & 13 & ParserJava.java \\ \hline
        Unit Tests  & Elisabeth S.  & 07.06.2019 & 17.06.2019  & 14 & *GeneratorTest.java \\ \hline
        Gui  & Julian U. & 07.06.2019 & 17.06.2019  & 8 & GUISwing.java \\ \hline
\end{longtable}     
\nsecend

\nsecbegin{Konkrete Code-Qualität im Sprint}
Durch das Entfernen der Codefragmente, die in nahezu jeder Klasse vorgenommen wurden hat das Projekt an Übersichtlichkeit gewonnen. In einigen Klassen (Konsole bspw.) wurden komplett redundante Codezeilen durch effizientere Lösungen ersetzt und die Wartbarkeit erhöht.
\nsecend%Konkrete Code-Qualität im Sprint

\nsecbegin{Konkrete Test-Überdeckung im Sprint}
Die Testüberdeckung ist mit 62,9\% leicht zurückgegangen, aber noch durchaus im vertretbaren Bereich.
\nsecend%Konkrete Test-Überdeckung im Sprint

\nsecbegin{Ergebnisse des Reviews}
\begin{table}[H]

\begin{tabularx}{\textwidth}{ |l|l|X| }
\hline
\textbf{Klasse} & \textbf{Methode} & \textbf{Anmerkungen}\\
 \hline
XmlHelperMethods & delNode & Löschen eines Knotens aus xml-Dokument, wurde implementiert und getestet\\ \hline
XmlHelperMethods & writeToFile & Generiert XML Datei, wurde implementiert und getestet\\ \hline
XmlHelperMethods & compareXml & Rückgabewert wurde geändert \\ \hline
Sequenzdiagramm & createDiagram & Bugfixes, wurde getestet\\\hline
Logger & LogMain & Auf xml-Struktur umgestellt, korrekt angepasst\\\hline
Gui & showGui & Auflistung der Klassen und Methoden, fehlt\\\hline
ParserJava & parse & Bugs wurden gefixt, weitere Anpassungen ausstehend \\ \hline
Console & interactiveMode & Anpassung auf neue Methoden, Dateiauswahl, Fehlerbehandlung Nutzereingabe, wurde implementiert und getestet. Redundante Zeilen entfernt, Klassenvariablen eingebunden. \\ \hline
ClassDiagramGenerator & createDiagram & Ausgabe von Variablen und Methoden erweitert, wurde implementiert. Boolsche Variablen showInstances, showVars, showMethods fehlen.\\ \hline
\end{tabularx}
\end{table}

Sonstiges:
\begin{itemize}
\item Testabdeckung weiter erhöhen
\item Präsentation (Plakat, Demo) vorbereiten
\item Dritte Zwischenstandspräsentation am 21.06.2019
\end{itemize}
\nsecend%Ergebnisse des Reviews

\nsecbegin{Abschließende Einschätzung des Product-Owners}
Insgesamt wurde das Produkt in diesem Sprint durch das Entfernen redundanter Codefragmente besser wartbar und durch die Behebung einzelner Bugs weiter funktionsfähig gemacht.
\nsecend%Abschließende Einschätzung des Product-Owners

\nsecbegin{Abschließende Einschätzung des Software-Architekten}
Die Sequenzdiagramme können für die Spezifikation über die Kommandozeile erzeugt werden. In der GUI scheint hier noch ein Fehler zu sein. Klassendiagramme für Java funktionieren zuverlässig. Für C++ noch nicht. 
\nsecend%Abschließende Einschätzung des Software-Architekten

\nsecbegin{Abschließende Einschätzung des Team-Managers}
Allgemein kann Sprint 7 als erfolgreicher und produktiver Sprint gewertet werden. Mit 10 geschlossenen Tickets wurde die Entwicklung zu einer stabilen und lauffähigen Version weiter vorangetrieben. Die Bedienbarkeit der Software ist verbessert worden und die Erzeugung von Sequenzdiagrammen ist gegeben. Es bestehen weiterhin kleinere Fehler im Parser und in bei der Erzeugung von Klassendiagrammen. Mit dem nächsten Sprint sollten diese Bugs allerdings behoben werden und die Software wird im vollen Maße einsatzbereit sein.
Positiv anzumerken ist die selbstständige Arbeitsweise aller Teilnehmer, sowie die Zuweisung von Tickets und Nutzung der zur Verfügung stehenden Kommunikationsmittel.
\nsecend%Abschließende Einschätzung des Team-Managers
\nsecend

\nsecbegin{Sprint 8}
\nsecbegin{Ziel des Sprints}

\nsecend

\nsecbegin{User-Stories des Sprint-Backlogs}

\nsecend % {User-Stories des Sprint-Backlogs}

\nsecbegin{Zeitliche Planung}

\nsecend%Zeitliche Planung

\nsecbegin{Liste der durchgeführten Meetings}

\nsecend%Liste der durchgeführten Meetings

\nsecbegin{Ergebnisse des Planning-Meetings}

\nsecend

\nsecbegin{Aufgewendete Arbeitszeit pro Person$+$Arbeitspaket}

\nsecend

\nsecbegin{Konkrete Code-Qualität im Sprint}
Hat sich zum vorhergehenden Sprint kaum verändert.
\nsecend%Konkrete Code-Qualität im Sprint

\nsecbegin{Konkrete Test-Überdeckung im Sprint}
Die Testüberdeckung ist mit 62,3\% leicht zurückgegangen, aber noch durchaus im vertretbaren Bereich.
\nsecend%Konkrete Test-Überdeckung im Sprint

\nsecbegin{Ergebnisse des Reviews}

\nsecend%Ergebnisse des Reviews

\nsecbegin{Abschließende Einschätzung des Product-Owners}
Insgesamt konnte das Produkt zu einem zufriedenstellenden Ausmaß finalisiert werden. Außer einzelner Bedienungseigenarten der GUI funktioniert das Erstellen von Klassen- und Sequenzdiagrammen für die definierten Testfälle intuitiv.
\nsecend%Abschließende Einschätzung des Product-Owners

\nsecbegin{Abschließende Einschätzung des Software-Architekten}
\begin{itemize}
\item Die nach Sprint 6 gesetzten Ziele wurden erreicht.
\item Der Logger ist weitestgehend integriert.
\item Die Kommandozeilenschnittstelle ist umfangreich.
\item Die GUI und das Betrachten der Diagramme darin funktioniert
\item Es sind aber noch einige kleinere Bugs und Unschönheiten vorhanden
\end{itemize}
\nsecend%Abschließende Einschätzung des Software-Architekten
\nsecend
%%%%%% weitere Sprints analog


\nsecbegin{Dokumentation}
\end{shownto} %{-, developer, manualDE}
\begin{shownto}{-, developer}

\nsecbegin{Handbuch}
\nsecbegin{Kommandozeilenaufruf}
Aufruf:\\
java -jar puml.jar [-c -i inputPathes -o outputPathes (-cc | -s | -int | -cs entryClass entryMethod) [-ijar, -ijava]]

\begin{itemize}
\item Wird das Programm ohne Parameter aufgerufen öffnet sich die grafische Oberfläche
\item -c wird verwendet um einen Konsolen-Aufruf zu initiieren
\item  Nach -c müssen mit -i die Eingabepfade, mit -o der Ausgabepfad und je nach gewünschter Aktion -cc, s, -int oder -cs stehen
\item mit -ijar und -ijava können bestimmte Dateitypen ignoriert werden
\item interaktiver Modus
\begin{itemize}
\item Fragt ab welcher Typ von Diagramm erzeugt werden soll
\item Listet bei einem Sequenzdiagramm alle Klassen und Methoden auf und ermöglicht die Auswahl des Einstiegspunktes
\end{itemize}
\end{itemize}

\begin{tabular}{|c|c|c|p{5cm}|}
\hline 
Parameter & Übergabewerte & Info & Beschreibung \\ 
\hline 
-c & & Konsolenaufruf & keine grafische Oberfläche \\ 
\hline 
-i & inputPathes & Input & durch Strichpunkt getrennte Liste der Pfade zu den einzulesenden Dateien und Ordnern \\ 
\hline 
-o & outputPath & Output & muss ein Pfad zu einem Ordner sein. Hier soll der plantUML-Code und das Diagramm erzeugt werden \\ 
\hline 
-ijar & & Ignore Jar & .jar-Dateien werden ignoriert \\ 
\hline 
-ijava & & Ignore Java & .java-Dateien werden ignoriert \\ 
\hline 
-cc & & create Classdiagram & erzeugt ein Klassendiagramm \\ 
\hline
-cs & entryClass, entryMethod & create Sequencediagram & erzeugt ein Sequenzdiagramm \\
-s & & show & listet alle Klassen und Methoden auf \\
-int & & interactive & interaktiver Modus \\
\hline 
\end{tabular} 
\nsecend %{Kommandozeilenaufruf}

\nsecbegin{Installationsanleitung}
\nsecbegin{Allgemein}
Folgende Schritte müssen durchgeführt werden:
\begin{itemize}
\item Java oder OpenJDK in Version 11 installieren
\item Graphviz installieren
\item jar-Datei entpacken
\end{itemize}
\nsecend

\nsecbegin{Linux (Ubuntu)}
In Kommandozeile ausführen:
\begin{itemize}
\item sudo apt-get install openjdk-11-jre graphviz unzip
\item unzip puml\_install.zip -d \textasciitilde{}/myInstallPath
\end{itemize}
\nsecend

\nsecbegin{Windows}
Es wird die Verwendung von OpenJDK-11 empfohlen.
\begin{itemize}
\item puml\_install.zip entpacken
\item OpenJDK-11 befindet sich in der zip-Datei. Installationsanleitung unter \url{https://stackoverflow.com/questions/52511778/how-to-install-openjdk-11-on-windows}
\item Graphviz-installation durch doppelklick auf "'graphviz-2.38.msi"' starten und durchführen 
\end{itemize}
\nsecend

\nsecend %{Installationsanleitung}
\nsecend %{Handbuch}

\end{shownto} %{-, developer}
\begin{shownto}{-, developer, manualDE}


\nsecbegin{Software-Lizenz}
\nsecbegin{Begründung der Entscheidung}
%Autor: Patrick Otte
Da die gesammte Ausgabe der Diagramme aus PUML auf der PlantUML Bibliothek basiert, welche selbst unter der GNU Public License (GPL) veröffentlicht wurde, kann PUML auch nur durch GPL lizensiert werden.
\\\\
Die Grundregel ergibt sich aus Ziffer 2b) GPLv2:
\\\\
\textit{You must cause any work that you distribute or publish, that in whole or in part contains or is derived from the Program or any part thereof, to be licensed as a whole at no charge to all third parties under the terms of this License.
}\\\\
Somit fiel die Wahl der Lizenz für das PUML-Programm auf die GNU Public License.
\nsecend

\nsecbegin{Lizenztext}
This program is free software: you can redistribute it and/or modify
it under the terms of the GNU General Public License as published by
the Free Software Foundation, either version 3 of the License, or
(at your option) any later version.
\\
This program is distributed in the hope that it will be useful,
but WITHOUT ANY WARRANTY; without even the implied warranty of
MERCHANTABILITY or FITNESS FOR A PARTICULAR PURPOSE.  See the
GNU General Public License for more details.
\\
You should have received a copy of the GNU General Public License
along with this program.  If not, see <http://www.gnu.org/licenses/>.
\\
Dieses Programm ist Freie Software: Sie können es unter den Bedingungen
der GNU General Public License, wie von der Free Software Foundation,
Version 3 der Lizenz oder (nach Ihrer Wahl) jeder neueren
veröffentlichten Version, weiter verteilen und/oder modifizieren.
\\
Dieses Programm wird in der Hoffnung bereitgestellt, dass es nützlich sein wird, jedoch
OHNE JEDE GEWÄHR,; sogar ohne die implizite
Gewähr der MARKTFÄHIGKEIT oder EIGNUNG FÜR EINEN BESTIMMTEN ZWECK.
Siehe die GNU General Public License für weitere Einzelheiten.

Sie sollten eine Kopie der GNU General Public License zusammen mit diesem
Programm erhalten haben. Wenn nicht, siehe <https://www.gnu.org/licenses/>.
\nsecend
\nsecend %{Software-Lizenz}
\nsecend %Dokumentation


\nsecbegin{Projektabschluss}

\nsecbegin{Protokoll der Abnahme und Inbetriebnahme beim Kunden}
\begin{itemize}
\item Abnahme am 11.07.2019
\item Kunde hatte Ubuntu16.04, daher musst OpenJDK von Hand nachinstalliert werden
\item Einige Diagramme konnten zur Demonstration der Funktionalität erzeugt werden
\end{itemize}
\nsecend

\nsecbegin{Präsentation auf der Messe}
\begin{figure}
	\centering
	\includegraphics{Bilder/Puml_Plakat.png}
	\caption{Poster PUML}
	\label{img:Poster}
\end{figure}

Die Präsentation beinhaltete das geforderte Poster und eine eigens erstellte Animation die per Beamer an die Wand projeziert wurde. Auf dem Poster konnte der Besucher einen ersten Einblick von den Funktionen der Software bekommen. Die Animation zeigte, ähnlich dem Plakat, die Funktionen der Software und bot einen Einblick in die Funktionsweise.
Des Weiteren wurden Süßigkeiten für die Besucher des Messestandes bereit gestellt.
\nsecend

\nsecbegin{Abschließende Einschätzung durch Product-Owner}
XXX
\nsecend

\nsecbegin{Abschließende Einschätzung durch Software-Architekt}
\nsecbegin{Funktionalität}
Die Kernfunktionalität wurde implementiert und funktioniert. Das Erzeugen der Klassendiagramme sollte für beliebigen Java-Code funktionieren. Das Erzeugen der Sequendiagramme funktioniert für die Java-Spezifikation. Hier fehlen noch folgende Sonderfälle:
\begin{itemize}
\item instanz.function(instanz.function)
\item instanz.functionMitInstAlsReturn.function()
\end{itemize}
Das Erzeugen der Klassendiagremme für C++-Code funktioniert ebenfalls für die Spezifikation. Allerdings fehlen auch hier diverse Sonderfälle.\\
Die GUI ist soweit vollständig implementiert, hat aber noch einige kleine Bugs. \\
Die Kommandozeilen-Schnittestelle ist recht umfangreich.\\
\nsecend
\nsecbegin{Tests}
Die Testüberdeckung liegt bei ca. 62\%. Wobei einige der Tests noch fehlschlagen. Größtenteils aber auch unwichtigen Gründen. Z.B. Vergleich von Bildern welche nicht Byte für Byte identisch sind.
\nsecend
\nsecbegin{Architektur}
Beim entwerfen der Architektur wurde darauf Wert gelegt, dass das Projekt modular aufgebaut ist. Dies vereinfachte zum einen die Verteilung der Aufgaben und ermöglicht zum anderen das einfache hinzufügen oder austauschen einzelner Module.
\nsecend
\nsecbegin{Team}
Das Team besteht wie bei Projektbeginn immer noch aus 10 Bachelor- und 2 Master-Studenten. Aufgrund der Größe des Teams fiel es uns als Projektleitung nicht immer leicht alle Teammitglieder mit Sinnvollen Aufgaben zu betrauen. Daher wurde auch viel im Pair-Programming gearbeitet. Dies funktionierte für die meisten Zweiergruppen sehr gut. Die Zusammenarbeit des Teams war von Anfang an gut, verbesserte sich aber im Laufe des Projektes noch deutlich. Insbesondere wurden immer mehr für die Entwicklung relevante Informationen direkt unter den Teammitgliedern propagiert und nur noch für die Projektleitung wichtige Informationen an diese herangetragen. Das Team-Umfeld habe ich als professionell und angenehm empfunden. Zumal auch meist die Stimmung sehr gut war. 
\nsecend
\nsecbegin{Fähigkeiten der Teammitglieder}
Während zu Beginn noch viele Teammitglieder mit den für sie neuen Technologien (Git, LaTeX, Linux als Betriebssystem, Eclipse als IDE, JUnit, ... später auch XML und XPath) zu kämpfen hatten, bereiteten diese gegen Ende des Projektes kaum mehr Probleme und es wurde sich stark auf die Weiterentwicklung des Quellcode konzentriert. Auch die Qualität des Quellcodes nahm beständig zu. Wobei hier aber noch durchaus Optimierungspotential besteht. Da man Programmieren aber meiner Ansicht nach größtenteils durch Praktische Erfahrung lernt und es Jahre braucht um es wirklich zu beherrschen, wäre alles andere Verwunderlich.
\nsecend
\nsecend %{Abschließende Einschätzung durch Software-Architekt}

\nsecbegin{Abschließende Einschätzung durch Team-Manager}
Gibt es nicht.
\nsecend


\nsecend %Projektabschluss

\nsecdocumentend

\end{shownto} %{-, developer, manualDE}

\end{document}
