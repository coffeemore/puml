\nsecbegin{Workshop-Bedarfsermittlung}
Denkanstöße:
\begin{itemize}
\item Was für Kompetenzen könnten dem Team für die Entwicklung des Produkts fehlen?
\end{itemize}
Um die Recherche hier zu vereinfachen, wäre es nicht schlecht, wenn die anderen Recherchierenden den Bearbeitenden auf mögliche Wissenslücken aufmerksam machen. Ziel dieser Recherche ist ein allgemeiner Überblick, wo Defizite vorhanden sind. Es ist zu vermuten, dass die anfänglich identifizierten Probleme eher abstrakter Natur sind - konkrete "Baustellen" zeigen sich für gewöhnlich erst in der Entwicklung. Erwartet werden also keine haargenauen Angaben zu Kompetenzen bzw. Inkompetenzen.
\\
Fehlende Kompetenzen für die Entwicklung des Produktes:
\begin{itemize}
\item Keine UML-Kenntnisse
\item Keine Programmierkenntnisse welche über die bisherigen Anforderungen des Studiums hinausgehen
\item Keine Erfahrungen mit größeren Gruppenarbeiten
\item Keine bisherige GUI-Programmierung mit Hilfe von Tools
\item Noch nie mit GIT gearbeitet
\item Keine Kenntnisse der Fähigkeiten der Gruppenmitglieder
\end{itemize}
\\
Genannte Stichpunkte treffen zwar meist nicht auf alle, aber dennoch auf einen Großteil der Gruppe zu.
Im Laufe des Projekts werden die meisten fehlenden Kompetenzen automatisch wegfallen, da diese auf Erfahrungen aufbauen. Das Arbeiten mit GIT zum Beispiel wird sicherlich nach mehreren Benutzen einfacher.
\nsecend
