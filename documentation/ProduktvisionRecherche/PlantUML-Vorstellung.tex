\nsecbegin{PlantUML-Vorstellung}

Diese Recherche soll einen kurzen, ersten Einblick in den Aufbau und die Möglichkeiten von PlantUML gewähren.\\
Beschreibung in PlantUML
\begin{itemize}
\item web server unter: http://www.plantuml.com/plantuml/ für kleine Test, zum ausprobieren

\item Syntax\\
Beginn mit: „@startuml“
Ende:  „@enduml“\\
Kein „;“ zur Abgrenzung\\
Neue Zeile für neue Anweisung\\
Bsp.:\\
\begin{lstlisting}
@startuml
Bob -> Alice
@enduml
\end{lstlisting}

\item Klassen\\
- Werden durch Schlüsselwort „class“ eingeleitet\\
- Kann Attribute und/oder Methoden erhalten\\
- Bei längeren Klassen „…lange Klasse…“ as long\\
- Sichtbarkeit der Klasse über Symbole Minus, Tilde, Plus und Doppelkreuz\\
- Attribute und Merkmale können „Abstract“ oder „Static“-Merkmale bekommen über geschweifte Klammern\\
\\
Bsp. 1 - Einzelne Klasse:\\
\begin{lstlisting}
@startuml
class ErsteKlasse
@enduml
\end{lstlisting}
Bsp. 2 - Klasse mit langem Namen:\\
\begin{lstlisting}
@startuml
class "wirklich alleralleraller ErsteKlasse" as long
@enduml
\end{lstlisting}
Bsp. 3 - Klasse mit Methode und Datendeklaration:\\
\begin{lstlisting}
@startuml
class ErsteKlasse : Name
@enduml
Bsp.:
@startuml
class ZweiteKlasse{
String name
Integer wert
void methode(String parameter1, Int parameter2)
}
@enduml
\end{lstlisting}
Bsp. 4 - Sichtbarkeit der Klassen:\\
\begin{lstlisting}
Bsp.:
@startuml
class MeineKlasse{
- Private
# Protected
~ PackagePrivate
+ Public
}
@enduml
\end{lstlisting}
Bsp. 5 -  Static / Abstract:\\
\begin{lstlisting}
@startuml
class MeineKlasse{
      {static} String password
      {abstract} void methods()
      }
@enduml
\end{lstlisting}
Bsp. 6 - Abstrakte Klasse\\
\begin{lstlisting}
@startuml
abstract class AbstrakteZweiteKlasse
@enduml
\end{lstlisting}
Bsp. 7 . Interface\\
\begin{lstlisting}
@startuml
interface schoenesInterface
@enduml
\end{lstlisting}
\item Klassen Verbinden\\
- einfach\\
- gerichtet in eine Richtung\\
- in Beide Richtungen gerichtet\\
Bsp. 1 - Einfache Verbindung\\
\begin{lstlisting}
@startuml
class ErsteKlasse
class ZweiteKlasse
ErsteKlasse -- ZweiteKlasse
\end{lstlisting}
Bsp. 2 - Gerichtete Verbindung\\
\begin{lstlisting}
@startuml
class ErsteKlasse
class ZweiteKlasse
ErsteKlasse --> ZweiteKlasse
@enduml
\end{lstlisting}
Bsp. 3 - In beide Richtungen gerichtet\\
\begin{lstlisting}
@startuml
class ErsteKlasse
class ZweiteKlasse
ErsteKlasse <--> ZweiteKlasse
@enduml
\end{lstlisting}
Bsp. 4 - Lose-Verbindung\\
\begin{lstlisting}
@startuml
class ErsteKlasse
class ZweiteKlasse
ErsteKlasse o-- ZweiteKlasse
@enduml
\end{lstlisting}
Bsp. 5 - Gebundene Klassen\\
\begin{lstlisting}
@startuml
class ErsteKlasse
class ZweiteKlasse
ErsteKlasse *-- ZweiteKlasse
@enduml
\end{lstlisting}
Bsp. 6 - Vererbung Unterlass/Oberklasse\\
\begin{lstlisting}
@startuml
class ErsteKlasse
class ZweiteKlasse
ErsteKlasse o--|> ZweiteKlasse
@enduml
\end{lstlisting}
Bsp. 7 - Notiz\\
\begin{lstlisting}
@startuml
class ErsteKlasse
class ZweiteKlasse
ErsteKlasse -- ZweiteKlasse
note ``Dies ist`` as notiz1
ErsteKlasse .. notiz1
\end{lstlisting}
\item Pfeile
- kurze Pfeile/ lange Pfeile (-/—) unterschied Anordnung\\
- für includes und extends (..)\\
Bsp. 1 - kurz\\
\begin{lstlisting}
@startuml
class ErsteKlasse
class ZweiteKlasse
ErsteKlasse -> ZweiteKlasse
@enduml
\end{lstlisting}
Bsp. 2 - lang\\
\begin{lstlisting}
@startuml
class ErsteKlasse
class ZweiteKlasse
ErsteKlasse --> ZweiteKlasse
@enduml
\end{lstlisting}
Bsp. 3 - gestrichelt\\
\begin{lstlisting}
@startuml
class ErsteKlasse
class ZweiteKlasse
ErsteKlasse..> ZweiteKlasse
@enduml
\end{lstlisting}
Bsp. 4 - include und extend Assoziation\\
\begin{lstlisting}
Bsp2. gestrichelt - include und extend Assoziation:
@startuml
(A) .> (B): <<include>>
(B) .> (C): <<extend>>
@enduml
\end{lstlisting}
Bsp. 5 - Weitere Kombinationen zum Test auf www.plantuml.com/plantuml/\\
\begin{lstlisting}
@startuml
Bob -> Alice: synchrone Nachricht von Bob an Alice
Bob ->> Alice: asynchrone Nachricht von Bob an Alice
Bob --> Alice: gestrichelte Linie als Antwortnachricht
Bob -\ Alice: Pfeilspitze ist nur oberhalb der Linie gezeichnet
Bob ->x Alice: verlorene Nachricht von Bob an Alice, x an Pfeilspitze
Bob \\- Alice: Pfeilspitze ist nur unterhalb angezeigt, jedoch offen
Bob //-- Alice: gestrichelte Linie mit offener Pfeilspitze oberhalb
Bob ->o Alice: Kreis am Ende der Pfeilspitze auf der Lebenslinie von Alice Bob <-> Alice: bidirektionaler Pfeil
Bob <->o Alice: bidirektionaler Pfeil mit Kreis am Ende der Pfeilspitze
@enduml
\end{lstlisting}
\item Beschriftungen der Verbindungen\\
- einfache Annotationen durch „:“\\
- mit Richtungsangabe\\
Bsp. 1 - Simple Beschriftung\\
\begin{lstlisting}
@startuml
class ErsteKlasse
class ZweiteKlasse
ErsteKlasse - ZweiteKlasse : Beschriftung mit Leerzeichen
@enduml
\end{lstlisting}
Bsp. 2 - Beschriftung mit Leserichtung\\
\begin{lstlisting}
@startuml
class ErsteKlasse
class ZweiteKlasse
ErsteKlasse - ZweiteKlasse : Beschriftung mit Leserichtung <
@enduml
\end{lstlisting}
\item Limitierung der Darstellung in PlantUML\\
-keine direkten Verzweigungen, immer an Element gekoppelt\\
- es können inkonsistente Zeichnungen entstehen\\
Bsp. 1 - Vererbung im Kreis:
\begin{lstlisting}
@startuml
class a
class b
a -|> b
b -|> a
note "Fehler" as notiz1 a .. notiz1
@enduml
\end{lstlisting}
\item UML Erweiterungen von Interesse\\
Unterstützung für: \\
- Java API (plantuml.jar)\\
- png from String -png from File -svg from String\\
- Command Line/Terminal\\
- Eclipse (Plug-In)\\
- LaTex\\
- Atom, GEdit, VIM, weitere txt-editoren...\\
\end{itemize}

\nsecend
