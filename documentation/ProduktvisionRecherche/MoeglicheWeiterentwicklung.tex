\nsecbegin{Mögliche Weiterentwicklung (Autor: Arndt Tore)}
Denkanstöße:
\begin{itemize}
%\item Wie können wir unser Produkt verbessern, sobald es die definierten Kernfunktionalitäten enthält?
%\item (Am besten mit der Person erörtern, die zu Konkurrenzprodukten recherchiert) Welche Features bietet die Konkurrenz, die wir nicht für die Kernfunktionalität festgelegt haben?
%\item Grob geschätzt: Wie nützlich sind diese Features und wie aufwendig wäre deren Implementierung?
	\item Einlesen von Projekten aus verschiedenen Programmiersprachen
	\begin{itemize}
		\item Beispielsweise:
		\begin{itemize}
			\item Python
			\item C++
			\item C\#
			\item etc.
		\end{itemize}
			\item Nützlichkeit:
		\begin{itemize}
			\item Erhöht Einsetzbarkeit des Tools enorm
			\item Erhöht die Anzahl der möglichen Nutzer des Tools
		\end{itemize}
			\item Umsetzbarkeit:
		\begin{itemize}
			\item Durch die XML-Schnittstelle muss alleinig der Parser angepasst werden
			\item Den Parser ist allerdings die aufwendigste Klasse des Programms
		\end{itemize}
	\end{itemize}
	\item Erstellung eines Eclipse Plugins
	\begin{itemize}
		\item Nützlichkeit:
		\begin{itemize}
			\item Erhöht Usebility enorm
			\item Erleichtert die Benutzung
			\item Vergrößert die wahrscheinliche Anzahl der Nutzer durch einfache Einbindung
		\end{itemize}
			\item Umsetzbarkeit:
		\begin{itemize}
			\item Umsetzung durch Umsetzung in Konkurenzprodukten erleichtert
			\item Wahrscheinlich in geplantem Zeitraum umsetzbar
		\end{itemize}
	\end{itemize}
	\item Anpassen des Quellcodes durch Änderungen im erstellten UML Diagramm
	\begin{itemize}
		\item Nützlichkeit:
		\begin{itemize}
			\item Erhöht die Nutzbarkeit des Tools als ein Gesamtprodukt durch komplette Funktionalität
			\item Gibt dem Tool eine flexiblere Nutzbarkeit
		\end{itemize}
			\item Umsetzbarkeit:
		\begin{itemize}
			\item In vorgegebener Zeit schwer realisierbar
			\item Mögliche Erweiterung als weiterführendes Projekt
		\end{itemize}
	\end{itemize}
	
\end{itemize} %endTag
%Hier wurde z.B. schon eine Implementierung als Eclipse-Plugin und Support für verschiedene Programmiersprachen angesprochen.
\nsecend