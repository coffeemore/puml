\nsecbegin{Kernfunktionalität (Autor: Patrick Otte)}
%Autor: Patrick Otte
\begin{itemize}
\item Einlesen einer Jar-Datei oder mehrerer Java Dateien
\item Analyse des Java-Source Codes und Identifikation seiner verbundenen Klassen sowie deren Verknüpfungen und Methoden
\item Möglichkeit der Ausgabe eines Klassendiagrams oder eines Sequenzdiagramms
\item Sequenzdiagramm:
\begin{itemize}
	\item Möglichkeit der Ausgabe eines Sequenzdiagramms
	\item Möglichkeit Aufrufe von Methoden im Sequenzdiagramm zu blockieren
\end{itemize}	
\item Klassendiagramm:
\begin{itemize}
	\item Möglichkeiten des Nutzers der Auswahl der Bestandteile in einem Klassendiagramm
	\item Möglichkeit der Voransicht des Klassendiagramms
\end{itemize}	
\item Überstützung für das Layout:
\begin{itemize}
	\item Layout muss automatisch konfigurierbar sein
	\item Layout muss die Möglichkeit haben manuell konfiguriert werden zu können
\end{itemize}	
\item Beide Diagrammarten sollen als String oder als Textdatei ausgegeben werden können
\end{itemize}

Aufgaben für den 1. Sprint:
\begin{itemize}
\item Erstellen eines ersten GUI’s
\item Einlesen der Dateien sowie deren Analyse
\item Ausgabe des Klassendiagramms
\end{itemize}

%Denkanstöße:
%\begin{itemize}
%\item Was soll das Programm auf jeden Fall leisten?
%\item Was sind bezogen auf die Funktionalität die Prioritäten im ersten Sprint?
%\end{itemize}
%Die Kernfunktionalität lässt sich weitgehend aus der Projektbeschreibung ableiten. Ergänzend dazu sollte überlegt werden, ob sich aus dieser Beschreibung noch ungeklärte Fragen ergeben. Die Recherche dient also der Formulierung eines klaren Anforderungsprofils für unser Produkt.
\nsecend
