\nsecbegin{Anwenderprofil}
Denkanstöße:
\begin{itemize}
\item Wer ist unsere Zielgruppe?


Die Zielgruppe des Projekts beinhaltet alle auf Java programmierenden Personen welche zudem mit UML-Diagrammen meist vorwiegend komplexere Softwareprojekte visualisieren wollen. Spezifischer ist das Programm jedoch an fortgeschrittene bis professionelle Programmierer gerichtet um Softwareprojekte möglichst einfach und nach Wünschen des Nutzers in PlantUML zu überführen.  


\item Worauf legt unsere Zielgruppe besonderen Wert?


Die Zielgruppe fordert einerseits eine einfache Schnittstelle zwischen Quellcode und PlantUML um leicht und bedienerfreundlich automatisch UML-Diagramme zu erzeugen und gegebenenfalls Veränderung an der automatisch erzeugten Visualisierungen vorzunehmen.


\item Was für einen Mehrwert bietet unser Produkt der Zielgruppe?


Der Mehrwert des Projekts liegt bei der Füllung der Lücke zwischen PlantUML und Java-Quelltext, sodass einzelne oder eine Menge von Jar-Dateien automatisch eingelesen werden und in Voransicht dargestellt so bearbeitet werden können, dass der Nutzer die Entscheidungskraft darüber hat, welche Bestandteile in ein Klassen-Diagramm mit eingehen sollen. Zudem hat der Nutzer die Möglichkeit Sequenzdiagramme aus einer oder mehrerer Methoden zu erzeugen. Auch hier wird dem Nutzer eine Einscheidungsgewalt gewährt, bei dieser er bestimmte Aufrufe blockieren und sich bei alternativen Pfaden für einen der selbigen Entscheiden kann. Somit wird eine Diagramm mit Klassen, Verknüpfungen und und use-Beziehungen nach Vorstellungen des Nutzers erstellt.
\end{itemize}
Hier geht es einerseits um eine klare, wenn auch triviale Einordnung, wer unser Produkt später nutzen soll. Andererseits soll erörtert werden, wie das Produkt unserer Zielgruppe das Leben leichter machen kann.
\nsecend

    
    
    


