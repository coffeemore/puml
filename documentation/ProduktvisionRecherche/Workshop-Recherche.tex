\nsecbegin{Workshop-Recherche}
Denkanstöße:
\begin{itemize}
\item Wie können wir die Teammitglieder effektiv und effizient auf einen Stand bringen?
\item Welche Ressourcen könnten dafür nützlich sein (Scripting-APIs, gute Tutorials etc.)?
\item Wie können wir die gefundenen Ressourcen so zur Verfügung stellen, dass jeder einfach darauf zugreifen kann?
\end{itemize}
Diese Recherche ist einerseits eng mit der Recherche "Workshop-Bedarfsermittlung" verbunden, es schadet also auf jeden Fall nicht, sich über deren Ergebnisse zu informieren. Aber auch unabhängig davon können schon erste Ideen entwickelt werden, wie das Team miteinander und voneinander lernen kann.

\nsecbegin {Terminabsprache/Mitteilen von Neuigkeiten}
Zur Terminabsprache eignet sich gut ein Messenger wie zB Telegram, wo das Team eine Gruppe hat, übe die Neuigkeiten und Termine schnell ausgetauscht werden können und sich gebündelt an einem Ort befinden. 
\nsecend
\nsecbegin {Gefundene Ressourcen zur Verfügung stellen}
Es wäre sinnvoll, gefundene Ressourcen wie zum Beispiel Dokumente über einsetzbare Technologien, Tutorials oder andere Hintergrundinformationen außerhalb vom Gruppenschat teilen zu können, das können wir über git machen. 
\nsecend


\nsecend


