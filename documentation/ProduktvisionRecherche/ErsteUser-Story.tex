\nsecbegin{Erste User-Story}
Denkanstöße:
\begin{itemize}
\item Stell dir vor, das Produkt ist gerade fertig entwickelt worden und deine Teamleiter wollen ein Plant-UML von deinem Quellcode sehen (PUMLception). Welchen Workflow erwartest du? Was machst du in welcher Reihenfolge?
\item Optional: Besprich dich mit dem / der Recherchierenden für die GUI-Anforderungen. Habt ihr unterschiedliche Vorstellungen vom Workflow und wenn ja, wie unterscheiden sie sich?
\item Wie würde für dich rein intuitiv das Graphical User Interface aussehen? Wenn du das GUI mit nur drei Knöpfen bauen müsstest, welche Funktionen würdest du ihnen zuweisen?
\end{itemize}
Workflow:
\begin{enumerate}
\item Wähle im Programm „Öffnen“ aus und suche im Explorer die Datei
\item Datei wird eingelesen und verarbeitet
\item Ich sehe das ganze UML Diagramm auf der einen Seite und den Quelltext auf der anderen
\item Im Quelltext kann ich einzelne Parts auswählen, wodurch kleinere UML Diagramme erstellt werden
\item Ich kann die Anordnung der dargestellten Elemente verändern
\item Nun besteht die Möglichkeit das Diagramm z.B. als png-Datei zu exportieren
\end{enumerate}

GUI:
\begin{itemize}
\item Ganz oben wären Button zum minimieren, maximieren und schließen. Darunter eine Menüleiste mit verschiedenen Optionen um Quelltext und UML-Diagramme zu öffnen und speichern.
\item Am linken Rand ist die Klassenstruktur, wie man sie z.B. in Eclipse hat dargestellt zur Koordination. Direkt daneben befindet sich ein Feld mit dem Quelltext, damit man sich direkt dort vergewissern kann, wie Klassen, die im UML-Diagramm dargestellt sind miteinander im Quelltext interagieren.
\item Auf der linken Seite ist das UML Diagramm dargestellt.
\item Die drei wichtigsten Knöpfe der GUI wären: Öffnen, Speichern und Diagramm bearbeiten
\end{itemize}




\nsecend