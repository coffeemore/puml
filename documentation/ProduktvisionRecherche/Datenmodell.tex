\nsecbegin{Datenmodell}
\begin{itemize}
\item Welche Daten verarbeiten wir?
\begin{itemize}
\item Quellcode, der vom Anwender gegeben wird
\item Daten sind abhängig vom Nutzer
\end{itemize}
\item Form der Datenaus- und -eingabe
\begin{itemize}
\item Eingabe: Jar-Datei oder mehrere Java-Dateien, die Quellcode enthalten oder Textdateien
\item Ausgabe: Klassendiagramme, Sequenzdiagramme, ... mit den Beziehungen zwischen den einzelnen Klassen
\end{itemize}
\item wichtige Variablen und Parameter
\begin{itemize}
\item Klassenname
\item Klassenattribute
\item Relationen zwischen den einzelnen Klassen (mit Pfeilen dargestellt)
\item Methoden der einzelnen Klassen
\item Eigenschaften wie abstract, private, protected, etc.
\end{itemize}
\item wichtige Klassen aus der Java-Standardbibliothek
\begin{itemize}
\item java.awt - zum Erstellen von User-Interfaces
\item jav.io - zum Einlesen von Dateien
\item java.util - enthält z.B. event model, frameworks, internationalization, ...
\item javax.swing - enthält Klassen zum Erstellen einer GUI
\end{itemize}
\end{itemize}
\nsecend