\begin{table}[H]

\begin{tabularx}{\textwidth}{ |l|X|X| }
  \hline
  \textbf{Datum} & \textbf{Anliegen oder Fragen} & \textbf{Ergebnisse}\\
  \hline
  \multirow{2}{*}{02.11.18} & Wie genau soll das Layout des Diagramms anpassbar sein? & Das Layout soll sowohl manuell als auch automatisch optimiert werden können. \\\cline{2-3}
  & Reicht es für den ersten Sprint, wenn PUML als Kommandozeilenprogramm umgesetzt wird? & Es soll möglichst früh eine grafische Oberfläche entwickelt werden. Deren Funktionsumfang darf zu Beginn ruhig minimal sein. Wichtig ist, dass das Team möglichst früh einen \glqq optischen Erfolg\grqq{} zu verzeichnen hat. \\ \hline
  \multirow{2}{*}{10.01.19} & Müssen auch vorkompilierte .jar-Dateien eingelesen werden können? & Eine Dekompilierung von bereits vorkompilierten .jar-Dateien ist nicht erforderlich, in diesem Fall reicht es, wenn eine Fehlermeldung ausgegeben wird, dass bereits kompilierte Dateien ausgewählt wurden. \\ \cline{2-3}
  & Wie schlimm ist es, dass das Tool für die grafische Oberfläche nicht unter Linux läuft? & Am wichtigsten ist später zwar die Plattformunabhängigkeit des Produkts, dennoch sollte nach Möglichkeit sichergestellt werden, dass jedes der Entwicklungswerkzeuge jedem Entwickler zur Verfügung steht, egal, auf welchem Betriebssystem entwickelt wird. \\
  \hline
\end{tabularx}
\end{table}
