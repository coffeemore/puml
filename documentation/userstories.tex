\nsecbegin{Userstories}
\nsecbegin{Vorschau}
Als Benutzer wünsche ich mir eine Vorschau der Diagramme, damit ich einschätzen kann ob ich damit zufrieden bin.
\nsecend

\nsecbegin{Interfaces}
Als Benutzer wünsche ich mir, dass ich abhängig von Interfaces die zugehörigen Klassen anzeigen lassen kann, damit ich weiß, welche Methoden ich implementieren muss.
\nsecend

\nsecbegin{Kommandozeile}
Als Benutzer wünsche ich mir, dass das Programm von der Kommandozeile aus aufrufbar ist, um es automatisiert starten zu können.
\nsecend

\nsecbegin{Dateien einlesen}
\nsecbegin{Art der eingelesenen Datei}
Als Benutzer wünsche ich mir, dass eine Auswahl zwischen Jar- und Java-Dateien möglich ist, damit Quellcode nicht doppelt eingelesen wird.
\nsecend

\nsecbegin{Java-Dateien}
Als Benutzer wünsche ich mir, dass Java-Dateien einlesbar sind, um den Quellcode von einer oder mehreren Klassen zu analysieren.
\nsecend

\nsecbegin{Jar-Dateien}
Als Benutzer wünsche ich mir, dass Jar-Dateien einlesbar sind, um den Quellcode zu analysieren.
\nsecend
\nsecend

\nsecbegin{Klassendiagramme}
Als Benutzer wünsche ich mir, Klassendiagramme aus meinem bestehenden Quellcode erstellen zu können, damit ich das nicht manuell tun muss.
\nsecend

\nsecbegin{Sequenzdiagramme}
Als Benutzer wünsche ich mir, Sequenzdiagramme aus meinem bestehenden Quellcode erstellen zu können, damit ich das nicht manuell tun muss.
\nsecend

\nsecbegin{Klassenauswahl}
Als Benutzer wünsche ich mir die Möglichkeit, Klassen zu selektieren, damit die Diagramme nicht zu unübersichtlich werden.
\nsecend

\nsecbegin{Methoden- und Variablenauswahl}
Als Benutzer wünsche ich mir die Möglichkeit, Methoden und Variablen zu selektieren, damit die Diagramme nicht zu unübersichtlich werden.
\nsecend

\nsecbegin{Multiple Klassenauswahl}
Als Benutzer wünsche ich mir einen Button, mit dem ich alle Klassen an- oder abwählen kann, damit ich nicht alle Klassen einzeln auswählen muss.
\nsecend

\nsecbegin{Multiple Methodenauswahl}
Als Benutzer wünsche ich mir einen Button, mit dem ich alle Methoden an- oder abwählen kann, damit ich nicht alle Methoden einzeln auswählen muss.
\nsecend

\nsecbegin{Layout}
Als Benutzer wünsche ich mir, das Layout meiner Diagramme ändern zu können, um deren Aussehen zu verbessern.
\nsecend


\nsecbegin{Drag-and-Drop}
Als Benutzer wünsche ich mir, die ausgewählten Dateien oder Ordner per Drag-and-Drop in das Programm aufzunehmen, damit ich den Datei-öffnen-Dialog nicht nutzen muss.
\nsecend

\nsecbegin{Anzeigen und Speichern von PlantUML}
Als Benutzer wünsche ich mir, Diagramme als PlantUML-Code anzeigen und speichern zu können, um den Aufbau nachvollziehen zu können.
\nsecend

\nsecbegin{Speichern von Bilddateien}
Als Benutzer wünsche ich mir, die erstellten Diagramme als Bilddatei exportieren zu können, um sie in meine Projektdokumentation mit aufzunehmen.
\nsecend

\nsecbegin{Speichern von Konfigurationen}
Als Benutzer wünsche ich mir, meine Konfiguration speichern zu können, damit ich meine Präferenz nicht jedes Mal aufs Neue einstellen muss.
\nsecend

\nsecbegin{Benutzerhandbuch}
Als Benutzer wünsche ich mir, das Benutzerhandbuch über die GUI anzeigen lassen zu können, damit ich die gedruckte Version nicht benötige.
\nsecend

\nsecbegin{Übersicht aller Diagramme für ein Projekt}
Als Benutzer wünsche ich mir eine Übersicht aller Diagramme, die ich für mein Projekt erstellt habe, damit ich leichter auf diese zugreifen kann.
\nsecend

\nsecbegin{Drucken}
Als Benutzer wünsche ich mir, Diagramme über das GUI drucken zu können, damit ich die Bilddateien nicht separat öffnen muss.
\nsecend

\nsecbegin{Exceptions als Sequenzdiagramme}
Als Benutzer wünsche ich mir, dass der mögliche Pfad der Exceptions als Sequenzdiagramm angezeigt werden kann, um ungehandelte Exceptions zu vermeiden.
\nsecend

\nsecend